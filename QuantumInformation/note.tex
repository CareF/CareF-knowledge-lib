%!TEX program = xelatex
%!TEX encoding = UTF-8 Unicode  
\documentclass[11pt,a4paper]{article}%titlepage表示标题单独页
\usepackage{xeCJK}
\setCJKmainfont{SimSun}
%\renewcommand{\thefigure}{\chinese{figure}}%将图片计数改为汉字数字
%\renewcommand{\thetable}{\chinese{table}}%将表格计数改为汉字数字
\usepackage[top=0.75in,bottom=0.75in,left=1in,right=1in]{geometry}%页边距设置:数据来自MS word的默认值
%\usepackage[CJKbookmarks]{hyperref}%给pdf文档添加互动式链接和书签

\usepackage{amsmath,amssymb,esint}%数学公式类宏包;最末为积分符号拓展
\allowdisplaybreaks[0]%允许多行公式间换页,用//*表示不允许换页
\numberwithin{equation}{section}
\usepackage{bm}%加粗(用于vector)
%\usepackage{textcomp}%符号包,不能用于数学模式,建议不要和SIunits混用
\usepackage[squaren]{SIunits}%科学单位包,可以用于数学模式(为了统一不要和textcomp混用),squaren选项消除和amssymb的冲突
\usepackage{graphicx}%插图宏包
%\usepackage{picinpar}%图文绕排
\usepackage{array}%表格宏包
%\usepackage{longtable}%长表格宏包
\usepackage{multirow}%多行合并的表格宏包
%\usepackage{booktabs}%表格线宏包
\usepackage{braket}
\usepackage{mathrsfs}%数学字体

%\usepackage[basic,box,gate,oldgate,ic,optics,physics]{circ}%电路图宏包
%\usepackage[normalem]{ulem}%下划线,删除线等宏包,参数表示不修改\emph{}格式
%\usepackage{mychemistry}%化学宏包,包含mhchem和chemfig
%\usepackage[symbol]{footmisc}%脚注拓展,选项表示用符号做脚注记号

\renewcommand*{\vec}[1]{\bm{#1}}%矢量的格式,这里是加粗
\newcommand{\dif}{\mathrm{d}}
\newcommand{\diff}{\,\mathrm{d}}
\newcommand\mi{\mathrm{i}}
\newcommand\e{\mathrm{e}}%定义数学模式中常用的正体字符
\newcommand\kk{\rangle\!\rangle}
\newcommand\bb{\langle\!\langle}
\newcommand{\Tr}{\operatorname{Tr}}
\newcommand{\St}{{\mathsf{St}}}
\newcommand{\rank}{{\mathsf{rank}}}
\newcommand{\Span}{{\mathsf{Span}}}

\begin{document}
\title{Note for Quantum Information}
\author{吕铭 Lyu Ming}
\maketitle
\section{Basic definitions and tools} % (fold)
\label{sec:basic_definitions_and_tools}
\begin{enumerate}
    \item Mutually unbiased bases:
    \begin{equation}
     \{\psi_m\}_{m=0}^d,\{\phi_n\}_{n=0}^d,\braket{\psi_m|\phi_n} = 1/d
    \end{equation}
    \item The Pavia notation (double ket notation):
    \begin{align}
        |\Psi\kk &= \sum_{m,n} \braket{m|\Psi|n}\ket{m}\ket{n} \\
        \bb\Phi|\Psi\kk &=\Tr [\Phi^\dagger\Psi]\\
        (A\otimes B)|\Psi\kk &= |A\Psi B^T\kk \\
        \big|\ket \alpha \bra \beta\big\rangle\!\big\rangle &= \ket\alpha\ket{\beta^*}\\
        \bra\alpha\bra\beta\Psi\kk &= \braket{\alpha|\Psi|\beta^*}
    \end{align}
    Unlike the Dirac notation, he notation is basis dependent.
    \item CHSH inequality (special case of Bell inequality)
    \begin{align}
        \omega_C &= \sum_\lambda p(\lambda)\left[\frac 14 \sum_{a,b\in\{0,1\}}\sum_{x,y\in\{0,1\}} (-)^{x+y+ab}p_A(x|a,\lambda)p_B(y|b,\lambda)\right] \\
        &\le \frac 14\sum_{a,b}(-)^{f_A(a)+f_B(b)+ab} \le \frac 12 \\
        \omega_Q &= \frac 14 \sum_{a,b}\sum_{x,y}(-)^{x+y+ab}\left|\langle x,\theta_a|\langle y,\tau_b|\Psi^+\rangle \right|^2 \\
        &= \frac 14\sum_{a,b}(-)^{ab}\cos (\theta_a - \tau_b)\le \frac 1{\sqrt 2}
    \end{align}
    \item Doing partial trace to get marginal states
    \item Bell states for $d$ dimensions: 
    \begin{equation}
        \ket{\Phi_{p,q}} := (S^pM^q\otimes I)\ket\Phi
    \end{equation}
    where $S = \sum \ket{(n+1)\mod d}\bra n$, and $M = \sum \exp (2\pi\mi n/d)\ket n\bra n$
    \item Trace-norm: Let $\Psi: \mathcal H_B \to \mathcal H_A$ and its sigular balue decomposition (SVD): $\Psi = \sum_n\lambda_n\ket{\alpha_n}\bra{\beta_n}$:
    \begin{equation}
        \Vert \Psi \Vert_1 := \sum_n|\lambda_n|
    \end{equation}
    Alternative characterization of the trace norm:
    \begin{equation}
        \Vert\Psi\Vert_1 = \max_{V:\mathcal H_A\to\mathcal H_B , V^\dagger V = I_A}\Tr[\Psi V]
    \end{equation}
    For pure states:
    \begin{equation}
        \Vert \pi_0\ket{\psi_0}\bra{\psi_0} - \pi_1\ket{\psi_1}\bra{\psi_1}\Vert_1 = \sqrt{1-4\pi_0\pi_1|\braket{\psi_0|\psi_1}|^2}
    \end{equation}
    $p$-norm:
    \begin{equation}
        \Vert \Psi \Vert_p := \left(\sum_n |\lambda_n|^p\right)^{1/p}
    \end{equation}
    \begin{itemize}
        \item $\forall c\in\mathbb C.\Vert\Psi\Vert = |c|\Vert\Psi\Vert$
        \item $\Vert\Psi\Vert = 0\Leftrightarrow \Psi = 0$
        \item $\Vert\Psi + \Psi'\Vert\le\Vert\Psi\Vert + \Vert\Psi'\Vert$
        \item $\Vert M\otimes N\Vert = \Vert M\Vert \Vert N\Vert$ 
    \end{itemize}
\end{enumerate}
% section basic_definitions_and_tools (end)

\section{General math model for Quantum computing} % (fold)
\label{sec:general_math_model_for_quantum_computing}
\begin{enumerate}
    \item Quantum state as density matrix $\rho$: 
    \begin{equation}
      \rho^\dagger = \rho, \quad \rho \ge 0,\quad \Tr[\rho] = 1
    \end{equation}
    The set of all density matrices is convex. The set of all pure states is the collection of all its extreme points.
    \item Quantum evolution as Quantum channel $\mathscr C(\rho)$:
    \begin{align}
      &\mbox{linear:}&&\mathscr C(\alpha\rho_1 + \beta \rho_2) = \alpha\mathscr C(\rho_1) + \beta\mathscr C(\rho_2),\\
      &\mbox{trace-preserving:}&&\Tr[\mathscr C(\rho)] = \Tr[\rho],\\
      &\mbox{completely positive:}&&\forall \mathcal H_B,\rho\ge 0. \mathscr C\otimes \mathscr I_B (\rho) \ge 0
    \end{align}
    \begin{enumerate}
        \item Physical implementation:
        \begin{equation}
          \mathscr C(\rho) = \Tr_{B'}\left[U_{AB\rightarrow A'B'}(\rho_A\otimes\sigma_B)U^\dagger_{AB\rightarrow A'B'}\right]
        \end{equation}
        \item Mathematical description (Kraus theorem):
        \begin{equation}
          \mathscr C(\rho) = \sum_k C_k \rho C_k^\dagger, \quad \mbox{where }\sum_kC_k^\dagger C_k = I_A
        \end{equation}
        \item isometric encoding: To encode the information carried by a system $A$ into a larger system $A'$.
        \begin{equation}
            \mathscr V(\rho) = V\rho V^\dagger
        \end{equation}
        where $V^\dagger V = I_A$
    \end{enumerate}
    \item Quantum measurement as POVM (\textit{positive operator-valued measure}) $\{P_n\}$:
    \begin{equation}
      p(n) := \Tr[P_n\rho],\quad P_n\ge 0,\quad \sum_n P_n = I
    \end{equation}
    \begin{enumerate}
        \item Physical implementation:
        \begin{equation}
          p(n) = \sum_k\braket{n|\mathscr C(\rho)|n} = \Tr\left[\sum_k C_k^\dagger\ket{n}\bra{n}C_k\rho\right]
        \end{equation}
        On the other hand:
        \begin{equation}
          \mathscr C(\rho) := \sum_n \Tr[P_n \rho] \ket n\bra n
        \end{equation}
        \item Example: describing not-accurate ONB measurement
        \begin{align}
            P_0 &= \int p(\theta)\ket{0,\theta}\bra{0,\theta}\diff\theta \\
            P_1 &= \int p(\theta)\ket{1,\theta}\bra{1,\theta}\diff\theta
        \end{align}
        \item Naimark's theorem: \\
        For every POVM $\{P_n\}$ there exists a system $B$ and a pure state $\ket\beta \in \mathcal H_B$ and a projective POVM $\{E_n\}$
        \begin{equation}
          \Tr[P_n\rho] = \Tr[E_n(\rho\otimes\ket\beta\bra\beta)]
        \end{equation}
    \end{enumerate}
    % \item Bayes rule for quantum states: Fot the state $\sigma_{AB}$ with measurement in system $B$: $\{Q_n\}$, the state of system $A$ conditional to the outcome $n$ is
    % \begin{equation}
    %     \sigma_{A|n} = \frac{\Tr_B [(I_A\otimes Q_n)\sigma_{AB}]}{\Tr [(I_A\otimes Q_n)\sigma_{AB}]}
    % \end{equation}
    \item Indirect measurement as Quantum Instrument $\{\mathscr Q_n(\rho)\}$: 
    \begin{align}
        &\mathscr Q = \sum_n \mathscr Q_n \mbox{ is a quantum channel}\\
        &p_n = \Tr [\mathscr Q_n(\rho)] \\
        &\rho_{|n} = \mathscr Q_n(\rho) / \Tr[\mathscr Q_n(\rho)] \mbox{ (Bayes rule for quantum states)}
    \end{align}
    where we also call $\mathscr Q_n$ quantum operation.
    \begin{enumerate}
        \item Physical implementation: 
        \begin{equation}
            \mathscr Q_n(\rho) := \Tr_B [(I_A\otimes Q_n)\mathscr C(\rho)]
        \end{equation}
        where we can define:
        \begin{align}
            \mathscr C (\rho) &:= \sum_n \mathscr Q_n(\rho)\otimes\ket n\bra n \\
            Q_n &:= \ket n\bra n
        \end{align}
    \end{enumerate}
\end{enumerate}
% section general_math_model_for_quantum_computing (end)

\section{Some mathematical operations} % (fold)
\label{sec:some_mathematical_operations}
\begin{enumerate}
    \item Purification:\\
    \begin{enumerate}
        \item The Schmidt decomposition:  
        \begin{align}
            \rho_A &= \sum_{m=1}^r p_m\ket{\alpha_m}\bra{\alpha_m} = \Tr_B[\sum_{m=1}^r\sqrt{p_m}\ket{\alpha_m}\ket{\beta_m}]
        \end{align}
        where $\{\ket{\alpha_m}\}$ and $\{\ket{\beta_m}\}$ are ONBs. The proof is an immediate consequence of the singular value decomposition (SVD).
        \item The uniqueness:
        \begin{align}
            &\rho_A = \Tr_A[\ket\Psi\bra\Psi] = \Tr_A[\ket{\Psi'}\bra{\Psi'}] \\
            \Rightarrow & \ket{\Psi'} = (I_A\otimes S)\ket\Psi
        \end{align}
        where $S$ is a partial isometry ($SS^\dagger$ and $S^\dagger S$ are projectors).
    \end{enumerate}
    \item Universal steering:\\
    Let $\ket\Psi\in\mathcal H_A\otimes\mathcal H_B$ be a purification of $\rho_A$, then for every decomposition $\rho_A = \sum_m p_m\rho_m$, there exists a POVM on $B$: $\{Q_m\}$:
    \begin{equation}
        \Tr_B[I_A\otimes Q_m\ket\Psi\bra\Psi] = p_m\rho_m
    \end{equation}
    \begin{itemize}
        \item From Bell state $\ket\Phi = \frac 1{\sqrt d}|I\kk$ to any state $\rho$ with POVM $\{\rho^T, I-\rho^T\}$:
        \begin{equation}
             \Tr_B[(I_A\otimes\rho^T)\ket\Phi\bra\Phi] = \frac\rho d
        \end{equation} 
    \end{itemize}
    \item Encoding a quantum operation in a quantum state: Choi matrix
    \begin{align}
        \Phi_{\mathscr M} &:= (\mathscr M\otimes \mathscr I)(|\Phi\kk\bb\Phi|) \\
        \mathscr M(\rho) &= d\Tr_B[(I_A\otimes \rho^T)\Phi_{\mathscr M}]
    \end{align}
    \begin{itemize}
        \item ``Diagonalize'' Kraus representation: $\mathscr C(\rho) = \sum_i C_i\rho C_i^\dagger$ with $\Tr[C_j^\dagger C_i] = \delta_{ij}p_id_A$
    \end{itemize}
    \item No information without disturbance: \\
    For the quantum instrument $\{\mathscr Q_n|\sum_n \mathscr Q_n = \mathscr I\}$ (that's the condition of non-disturbing), the outcome does not depend on measured system.
    \begin{equation}
        \sum_n\Phi_{\mathscr Q_n} = |\Phi\kk\bb\Phi| \Rightarrow \mathscr Q_n = p_n\mathscr I_n
    \end{equation}
    \item The no-cloning theorem: \\
    For two distinct non-orthogonal states $\ket{\varphi_0}$ and $\ket{\varphi_1}$, there is no quantum channel such that:
    \begin{equation}
        \forall i\in\{0,1\}. \mathscr C (\ket{\varphi_i}\bra{\varphi_i}) = (\ket{\varphi_i}\bra{\varphi_i})^{\otimes 2}
    \end{equation}
    What we still can do:
    \begin{itemize}
        \item Cloning orthogonal states
        \item The universal cloning machine
        \begin{equation}
            \mathscr C(\rho) = \frac 1{2(d+1)}(I + \mathtt{SWAP})(\rho\otimes I_B)(I + \mathtt{SWAP})
        \end{equation}
        \item Probabilistic cloning?..
    \end{itemize}
    Corollary:
    \begin{itemize}
        \item No-distinguishability theorem: It is impossible to construct a machine that distinguishes perfectly between two non-orthogonal states $\ket{\psi_0}$ and $\ket{\psi_1}$.
        \item Secure key distribution: the BB84 protocol (Bennett and Brassard, 1984).
    \end{itemize}
    \item Quantum teleportation (discribed as a quantum instrument)
    \begin{equation}
        \Tr_{A'A}[(I_B\otimes P_n)(\ket{\Phi_0}\bra{\Phi_0}_{BA'}\otimes\rho_A)] = \frac 14 U_n^\dagger\rho_A U_n
    \end{equation}
    where $\ket{\Phi_0} = (\ket{00}+\ket{11})/\sqrt 2$, $\ket{\Phi_i} = |\sigma_i\kk/\sqrt 2$ are Bell states, $U_0 = I$, $U_i = \sigma_i$, $P_n = \ket{\Phi_n}\bra{\Phi_n}$, means keep the state but transform it to another system. For higher dimension system:
    \begin{equation}
        \Tr_{AA'}[(\ket{\Phi_{pq}}\bra{\Phi_{pq}}_{AA'}\otimes I_B)(\rho_A\otimes\ket{\Phi_{00}}\bra{\Phi_{00}}_{A'B})] = \frac 1{d^2} U_{pq}^\dagger \rho_B U_{pq}
    \end{equation}
\end{enumerate}
% section some_mathematical_operations (end)

\section{Quantum discrimination}
    \subsection{State discrimination} % (fold)
    \label{sub:state_discrimination}
    \begin{enumerate}
        \item Helstrom's minimum error decoder:
        \begin{equation}
            \omega := p_{\mbox{succ}} - p_{\mbox{err}} = \sum_{x,y\in\{0,1\}} (-)^{x+y}\Tr [P_y\rho_x]\pi_x \le \Vert \Delta\Vert_1 := \sum|\delta_n|
        \end{equation}
        where $\delta_n$ is the eigenvalues of $\Delta = \pi_0\rho_0 - \pi_1\rho_1 = \sum_n\delta_n\ket{\psi_n}\bra{\psi_n}$. We reach the upper bound by:
        \begin{align*}
            P_0 &= \sum_{\delta_n > 0} \ket{\psi_n}\bra{\psi_n}\\
            P_1 &= \sum_{\delta_n \le 0} \ket{\psi_n}\bra{\psi_n}
        \end{align*}
        \begin{itemize}
            \item For pure state $\rho_0 = \ket{\psi_0}\bra{\psi_0}$, $\rho_1 = \ket{\psi_1}\bra{\psi_1}$, 
            \begin{equation}
                \omega_{\mbox{max}} = \sqrt{1-4\pi_0\pi_1 F},\quad F:= |\braket{\psi_0|\psi_1}|^2
            \end{equation}
            \item Fidelity for mixed states:
            \begin{equation}
                1-\sqrt{4\pi_0\pi_1F(\rho_0,\rho_1)}\le \Vert\pi_0\rho_0 - \pi_1\rho_1\Vert_1\le\sqrt{1-4\pi_0\pi_1F(\rho_0,\rho_1)}
            \end{equation}
            where:
            \begin{equation}
                F(\rho_0,\rho_1) := \sup_{\mathcal H_A} \max_{\Tr_A[\ket{\Psi_0}\bra{\Psi_0}] = \rho_0} \max_{\Tr_A[\ket{\Psi_1}\bra{\Psi_1}] = \rho_1} |\braket{\Psi_0|\Psi_1}|^2
            \end{equation}
            \item Uhlmann's theorem
            \begin{equation}
                F(\rho_0,\rho_1) = \Vert \sqrt\rho_0\sqrt\rho_1\Vert_1^2
            \end{equation}
            \item Minimum of the Bhattacharya coefficient:
            \begin{equation}
                F(\rho_0,\rho_1) = \min_N \min_{\forall \mbox{POVM}\{P_n\}_{n=1}^N}\left(\sum_n \sqrt{\Tr[P_n\rho_0]\Tr[P_n\rho_1]}\right)^2
            \end{equation}
            \item Quantum Chernoff bound: Error probability in distinguishing two states with $N$ copies goes to zero at rate $O(C^N)$ where 
            \begin{equation}
                C = \min_{p:0\le p\le 1}\Tr[\rho_0^p\rho_1^{1-p}] (<\sqrt{F(\rho_0,\rho_1)})
            \end{equation}
        \end{itemize}
        \item The unambiguous state discriminator: to distinguish $\{\ket{\psi_i}\}$, we use POVM $\{ P_i, P_?\}$, where we get answer without error or we don't know about the answer:
        \begin{align}
            \braket{\psi_i|P_j|\psi_i} = p_i\delta_{ij}\\
            P_? = I - \sum_i P_i
        \end{align}
        It is possible if and only if $\{\ket{\psi_n}\}_{n=1}^N$ are linearly independent.
        \begin{align}
            P_n &= p\Phi^{-1}\ket{\psi_n}\bra{\psi_n}\Phi^{-1}\\
            \Phi&:= \sum_n\ket{\psi_n}\bra{\psi_n}
        \end{align}
        For $N=2$ system $p_? = \sqrt F$
    \end{enumerate}
    % subsection state_discrimination (end)

    \subsection{Channel discrimination} % (fold)
    \label{sub:channel_discrimination}
    \begin{enumerate}
        \item Input any state:
        \begin{equation} 
         \omega_{\mathrm{max}} = \max_{\ket\alpha\in\mathcal H_A}\Vert \pi_0\mathscr C_0(\ket\alpha\bra\alpha) - \pi_1\mathscr C_1(\ket\alpha\bra\alpha)\Vert_1
        \end{equation}
        \item Input an entangled state:
        \begin{equation}
         \omega_{\mathrm{max}}^{\mathrm{ent}} = \max_{\mathcal H_B}\max_{\ket\Psi\in\mathcal H_A\otimes \mathcal H_B}\Vert \pi_0(\mathscr C_0\otimes\mathscr I_B)(\ket\Psi\bra\Psi) - \pi_1(\mathscr C_1\otimes\mathscr I_B)(\ket\Psi\bra\Psi)\Vert_1
        \end{equation}
        \item diamond norm
        \begin{equation}
            \Vert\Delta\Vert_\diamond = \max_{\ket\Psi\in\mathcal H_A\otimes \mathcal H_A}\Vert \Delta\otimes\mathscr I_A(\ket\Psi\bra\Psi)\Vert_1
        \end{equation}
        \item For unitary operator $U_0$, $U_1$. For eigenvalues $\e^{\mi\theta_m}$ of $U_0^\dagger U_1$: 
        \begin{equation}
            \omega = \sqrt{1-4\pi_0\pi_1F} = \sqrt{1-4\pi_0\pi_1\left|\sum_mp_m\e^{\mi\theta_m}\right|}
        \end{equation}
        \begin{itemize}
            \item Entanglement does not help
            \item Certainty answer can be get within finite number of times.
            \item Extend to more gates
        \end{itemize}
    \end{enumerate}
    % subsection channel_discrimination (end)

\section{Quantum programming} % (fold)
\label{sec:quantum_programming}
\begin{enumerate}
    \item programmable machine:
    \begin{equation}
     V\ket\alpha\ket n = U_n\ket\alpha\ket n
    \end{equation}
    Example: $V = \sum_n U_n\otimes\ket n\bra n
    $
    \item No-programming theorem (Nielsen-Chuang, PRL 1997): In order to program $N$ distinct unitary gates, one needs N orthogonal program states.
    \item Universal set of quantum gates: Every qubit gate can be approximated with arbitrary precision with a circuit consisting only of 2 elementary gates. And for a system in dimension $d\ge 2$, it is enough to use $O(\log^2d)$ gates.
    \begin{itemize}
        \item For $N$ qubits system (dimension $2^N$), It is enough to have a universal set for every qubit and a entangling gate $W_{ij}$ on every two qubits.
        \item Usually we use $\{H,T,\mathtt{CNOT}\}$:
        \begin{align}
            H &= \frac{1}{\sqrt{2}}\begin{pmatrix}
            1 &1 \\
            1 &-1\end{pmatrix}\\
            &= \mi\exp\left[\frac{-\mi\pi\vec n\cdot\vec\sigma}2\right] \qquad \vec n = \frac1{\sqrt2}(1,0,1)^T \\
            T &= \exp\left[\frac{-\mi\pi\sigma_z}8\right]\\
            \mathtt{CNOT} &= \ket0\bra0\otimes I + \ket1\bra1\otimes\sigma_x
        \end{align}
        \item Solovay-Kitaev's Theorem: Let $\mathtt U$ be a universal set of unitary gates in dimension $d$ with the property that $\forall U\in\mathtt U.\quad U^\dagger\in\mathtt U$. Then, every unitary gate in dimension $d$ can be approximated within an error $\epsilon$ as a product of $N$ gates in $\mathtt U$, where $N\sim O(-\log^c\epsilon)$, where $0<c<2$ is a suitable constant.
        \item No theory about $O(\log^\alpha d)$.. 
    \end{itemize}
\end{enumerate}
% section quantum_programming (end)

\section{Quantum Error Correction} % (fold)
\label{sec:quantum_error_correction}
\begin{enumerate}
    \item Basic steps:
    \begin{enumerate}
        \item Encoding : $\mathscr V(\rho) = V\rho V^\dagger$ (with isometry $V: \mathcal H_A\mapsto \mathcal H_{A'}, V^\dagger V = I$)
        \item Error: a quantum channel $\mathscr E: \mathcal H_{A'}\mapsto \mathcal H_{A'}$
        \item Measurement: a quantum instrument $\{\mathscr Q_i\}$
        \item Recovery: a unitary gate $U_i$ according to the outcome
        \item Decoding: $\mathscr D$
    \end{enumerate}
    The recovery channel $\mathscr R$: the last three steps together, $\mathscr R(\rho) := \sum_i\mathscr D\left(U_i\mathscr Q_i(\rho)U_i^\dagger\right)$\\
    Therefore here we require:
    \begin{equation}
        \mathscr{REV} = \mathscr I_A
    \end{equation}
    \item Definitions:
    \begin{itemize}
        \item A quantum channel $\mathscr C:\mathcal H_A\mapsto\mathcal H_{A'}$ is correctable iff $\exists \mathscr R.\quad \mathscr{RC} = \mathscr I$
    \end{itemize}
    \item Knill-Laflamme (KL) condition: A channel $\mathscr C(\rho) = \sum_i C_i\rho C_i^\dagger$ is correctable iff:
    \begin{equation}
        C_j^\dagger C_i = \sigma_{ij} I_a
    \end{equation}
    where $\sigma\in\St(\mathcal H)$. 
    \begin{enumerate}
        \item if $A = A'$, $\mathscr C$ is unitary, i.e. $\mathscr C(\rho) = U\rho U^\dagger$ 
        \item KL condition is equivalent to $\mathscr C(\rho) = \sum_m p_m V_m\rho V_m$ where $V_m^\dagger V_n = \delta_{mn}I_A$. That means the correctable channels are those that encode randomly the state into different \textbf{orthogonal subspaces}.
        \item Correction: measurement with $\{\mathscr Q_m\}$ and corresponding recovery $\mathscr R_m$:
        \begin{align}
            \mathscr Q_m(\rho) &= P_m\rho P_m, \qquad (P_m = V_m V_m^\dagger, P_0 = I_{A'} - \sum P_m )\\
            \mathscr R_m(\rho) &= V_m^\dagger \rho V_m + \Tr[(I_{A'} - V_mV_m^\dagger)\rho]\ket0\bra0
        \end{align}
        \item physical meaning of $\sigma$: If we generally define the channel $\mathscr C(\rho) = \Tr_B[W\rho W^\dagger]$ and its \emph{complementary channel} $\tilde{\mathscr C}(\rho) = \Tr_{A'}[W\rho W^\dagger]$, we have
        \begin{equation}
            \forall \rho\in\St(\mathcal H_A).\qquad \tilde{\mathscr C}(\rho) = \sigma
        \end{equation}
        which means that a channel is correctable iff its complementary channel is an erasure channel.
        \item KL condition for good codes: Let error $\mathscr E(\rho) = \sum_i E_i\rho E_i^\dagger$ ans the subspace $\mathcal S$ is a good code for $\mathscr E$ iff
        \begin{equation}
            PE_j^\dagger E_i P = \sigma_ij P
        \end{equation}
        where $\sigma\in\St(\mathcal H)$ ans $P$ is a projector on $\mathcal S$
    \end{enumerate}
    \item Quantum packing bound: $d_{A'}\ge d_A\rank(\sigma) =d_A\rank(\Phi_{\mathscr C})$
    \item Quantum packing bound non-degenerate codes: if given an orthogonal Kraus representation for $\mathscr E(\rho = \sum_i^k E_i\rho E_i^\dagger$, than $d_{A'}\ge d_Ak$. In principle degenerate code could probably do better.
    \item the quantum Hamming bound for arbitrary Pauli errors:
    \begin{equation}
        \mathscr E(\rho) =(1-p)\rho + \frac pt\sum_{m=1}^t\frac {m!(N-m)!}{N!3^m}\sum_{\vec n}\sum_{\vec k}\mathscr U_{\vec n,\vec k}(\rho)
    \end{equation}
    where $\vec n = (n_1,\cdots,n_m)$ labels $m$ qubits affected and $\vec k = (k_1,\cdots, k_m)$ the Pauli matrix acted. To encode $K$ qubits into $N$ qubits, the quantum packing bound for non-degenerate codes gives
    \begin{equation}
        2^{N-K} \ge \sum_{m=0}^t\frac{N!3^m}{(N-m)!m!}
    \end{equation}
    Whether there is better code is an open question.
    \item Correct one to correct them all: Let two channel $\mathscr C(\rho) = \sum C_i\rho C_i^\dagger$ and $\mathscr D(\rho) = \sum D_j \rho D_j^\dagger$ with $D_j\in\Span\{C_i\}$, and if $\mathscr C$ is correctable, $\mathscr D$ is also correctable with same recovery channel and good code subspaces.
    \begin{itemize}
        \item Specially for arbitrary Pauli errors
        \begin{equation}
             \mathscr E(\rho) = (1-p)\rho + \frac p{3N}\sum_{n=1}^N\sum_{k=1}^3\mathscr U_{n,k}(\rho)
         \end{equation} 
        where $\mathscr U_{n,k}$ is $\sigma_k$ applied on $n$-th qubit. A good code for $\mathscr E$ is a good code for any quantum channel acting on a single qubit, even erasure channel.
    \end{itemize}
\end{enumerate}
% section quantum_error_correction (end)

\section{Quantum entropy} % (fold)
\label{sec:quantum_entropy}
\begin{enumerate}
    \item LOCC protocol and one-way LOCC protocol %补充定义
    \item Lo-Popescu theorem: LOCC protocol and one-way LOCC protocol are equivalent.
    \item $\ket\Psi$ is more entangled than $\ket{\Psi'}$ iff there exists a LOCC channel that transforms $\ket{\Psi}$ into $\ket{\Psi'}$
    \begin{itemize}
        \item A product state is less entangled than any other bipartite state.
        \item Bell states is more entangled than any other bipartite state.
    \end{itemize}
    \item $\rho\in\St(\mathcal H)$ is more mixed than $\rho'\in\St(\mathcal H)$ iff $\rho$ can be obtained by applying a \emph{random-unitary (RU) channel}:
    \begin{equation}
         \rho = \sum_i p_iU_i\rho' U_i^\dagger
     \end{equation} 
    \begin{itemize}
        \item Every state $\rho$ is more mixed than a \textbf{pure state} $\rho' = \ket\psi\bra\psi$
        \item No state $\rho$ is more mixed than the state $\rho' = I/d$ (\textbf{maximally mixed state}), $\rho'$ is more mixed than any other state
    \end{itemize}
    \item Let $\ket\Psi,\ket{\Psi'}\in\mathcal H_A\otimes\mathcal H_B$ be two \emph{pure} bipartite states, the following are equivalent
    \begin{itemize}
        \item $\ket\Psi$ is more entangled
        \item the marginal of $\ket\Psi$ is more mixed
    \end{itemize}
    \item Generalization: Let $\rho\in\St(\mathcal H_A)$ and $\rho'\in\St(\mathcal H_{A'})$, $\rho$ is more mixed than $\rho'$ iff $\rho\otimes\ket{\alpha'}_{A'}\bra{\alpha'}_{A'}$ is more mixed than $\ket{\alpha}_A\bra\alpha_A\otimes\rho'$
    \item The majorization criterion: Let $\rho = \sum_i p_{i}\ket{\alpha_{i}}\bra{\alpha_{i}}$ with $p_{1}\ge p_{2}\ge\cdots\ge p_{d}\ge 0$, than $\rho$ is more mixed than $\rho'$ iff $\vec p$ is majorized by $\vec p'$ ($\vec p \preceq\vec p'$)
    \begin{equation}
        \forall k\in\{1,\cdots,d-1\},\qquad \sum_{i=1}^k p_{i}\ge\sum_{i=1}^k p'_{i}
    \end{equation}
    \item Measurement of mixedness: Schur-concave function %定义?
    \item R\'enyi entropies: a group of Schur-concave functions
    \begin{equation}
        H_\alpha = \frac 1{1-\alpha}\log\left[\sum_{i=1}^dp_i^\alpha\right]\qquad \alpha\ge 0
    \end{equation}
    And quantum R\'enyi entropies\footnote{Here for convenience to discuss bits and qubits, the $\log\cdot$ means $\log_2\cdot$.}
    \begin{equation}
        S_\alpha = \frac{\alpha}{1-\alpha}\log\Vert\rho\Vert_\alpha
    \end{equation}
    \begin{itemize}
        \item $\forall\alpha.\quad S_\alpha(\rho) = 0\Leftrightarrow \rank\rho = 1$
        \item $\forall\alpha>0.\quad S_\alpha = \log d\Leftrightarrow \rho = I/d$
        \item $\forall\alpha.\quad 0\le S_\alpha(\rho)\le\log d$
        \item Additivity property: $S_\alpha(\rho\otimes\sigma) = S_\alpha(\rho) + S_\alpha(\sigma)$
    \end{itemize}
    Special values of $\alpha$:
    \begin{enumerate}
        \item $\alpha = 0$, max-entropy
        \begin{equation}
             S_0(\rho) = \log[\rank(\rho)]
         \end{equation} 
        \item $\alpha\to\infty$, min-entropy
        \begin{equation}
            S_\infty(\rho) = -\log p_1
        \end{equation}
        \item $\alpha\to 1$ classically Shannon entropy, and quantumly von-Neumann entropy
        \begin{equation}
            S(\rho):=\lim_{\alpha\to 1}S_\alpha(\rho) = -\Tr[\rho\log\rho]
        \end{equation}
    \end{enumerate}
    \item Asymptotic transformations
    \begin{enumerate}
        \item A rate $R$ is achievable if for every $N$ there exists a LOCC channel $\{\mathscr L_N\}_{N\in\mathbb N}$
        \begin{equation}
            \lim_{N\to\infty}\Vert\mathscr L_N((\ket\Psi\bra\Psi)^{\otimes N})-(\ket{\Psi'}\bra{\Psi'})^{\otimes RN}\Vert_1 = 0
        \end{equation}
        \item Achievable rates and von-Neumann entropy
        \begin{equation}
            \sup\{R|R\mbox{ is achievable}\} = \frac{S(\rho)}{S(\rho')}
        \end{equation}
    \end{enumerate}
    \item *Concurrence: measurement of entanglement, PhysRevLett.78.5022 \\
    For two qubit system with states$\ket\psi = \sum_i\alpha_i\ket{\Phi_i}$ with Bell bases defined as 
    \begin{align}
        \ket{\Phi_1} &= (\ket{00}+\ket{11})/\sqrt2 \\
        \ket{\Phi_2} &= \mi(\ket{00}-\ket{11})/\sqrt2 \\
        \ket{\Phi_3} &= \mi(\ket{01}+\ket{10})/\sqrt2 \\
        \ket{\Phi_4} &= (\ket{01}-\ket{10})/\sqrt2
    \end{align}
    Calculate the von-Neumann entropy of the marginal we get the measurement of its entanglement
    \begin{equation}
        E(\psi) = \mathcal E(C(\psi))
    \end{equation}
    with $\mathcal E(x) := H(\frac 12+\frac 12\sqrt{1-x^2})$, $H(x) := -[x\log x + (1-x)\log (1-x)]$, and the concurrence $C(\psi)$ monotonical to $E$
    \begin{equation}
        C(\phi) = \left|\sum_i \alpha_i^2\right|
    \end{equation}
\end{enumerate}
\subsection{Quantum data compression} % (fold)
\label{sub:quantum_data_compression}
\begin{enumerate}
    \item Encoding channel $\mathscr E: \mathcal H_A \mapsto \mathcal H_B$ with $d_B<d_A$ and decoding channel $\mathscr D:\mathcal H_B \mapsto\mathcal H_A$ so that the average fidelity 
    \begin{equation}
        F = \sum_x p_x\braket{\psi_x|\mathscr D\mathscr E(\ket{\psi_x}\bra{\psi_x})|\psi_x}
    \end{equation}
    is close to $1$.
    \item Compressing entanglement:
    \begin{equation}
        F_{\mathrm{ent}} = \braket{\Psi|(\mathscr D\mathscr E\otimes\mathscr I_R)(\ket\Psi\bra\Psi)|\Psi}\le F
    \end{equation}
    \item Subspace encodings:
    \begin{equation}
        \mathscr E(\rho) = P\rho P + \Tr[(I-P)\rho]\ket{\psi_0}\bra{\psi_0}
    \end{equation}
    where $P$ is the projector on subspace $\mathcal S$ and $\ket{\psi_0}$ is in the orthogonal complement of $\mathcal S$
    \begin{equation}
        F_{\mathrm{ent}}\ge\left|\sum_xp_x\braket{\psi_x|P|\psi_x}\right|^2 := p_{\mathrm{yes}}^2
    \end{equation}
    \item asymptotic scenario: compression $\rho^{\otimes N}$ into $\mathcal S_N$ with dimension $d_N$, define the compression rate
    \begin{equation}
        R:=\limsup_{N\to\infty}\frac{\log d_N}N
    \end{equation}
    And achievable rate requires a sequence of coding so that
    \begin{equation}
        \limsup_{N\to\infty}F_N = 1
    \end{equation}
    \begin{enumerate}
        \item Define the description of $\rho^{\otimes N}$:
        \begin{equation}
            \rho^{\otimes N} = \sum_{\vec m}q_N(\vec m)\ket{\psi_{\vec m}}\bra{\psi_{\vec m}}
        \end{equation}
        where
        \begin{itemize}
            \item $\vec m = (m_1,\cdots, m_N)\in\{1,\cdots,d_A\}^{\times N}$
            \item $q_N(\vec m)$ is the probability $q_N(\vec m) = \prod_i q_{m_i}$ 
            \item $\ket{\psi_{\vec m}}$ is the product vector $\ket{\psi_{\vec m}} := \ket{\psi_{m_1}}\ket{\psi_{m_2}}\cdots\ket{\psi_{m_N}}$
        \end{itemize}
        \item the type of a sequence $\vec m$ defined by $t_{\vec m}:=(N_1/N,\cdots,N_{d}/N)$
        \begin{enumerate}
            \item total number of types: $T_N \sim 1$
            \item the number of sequences of type $t$: 
            \begin{equation}
                S_{N,t} \sim \exp[N H(t)]
            \end{equation}
            where $H(t_{\vec m}) := -\sum_{i=1}^d t_i\ln t_i$ is the Shannon entropy
            \item the probability that a sequence is of type $t$:
            \begin{equation}
                 Q_{N,t}\sim \exp[-ND(t\Vert q)]
            \end{equation}
            where $D(t\Vert q) := \sum_{i=1}^d t_i\ln\frac{t_i}{q_i}$ is the Kullback-Leibler divergence
        \end{enumerate}
        Kullback-Leibler divergence satisfies the following properties
        \begin{itemize}
            \item $D(t\Vert q)\ge 0$ and the equality holds iff $t = q $
            \item if $\lim_{N\to\infty}D(t_N\Vert q) =0$, than $\lim_{N\to\infty}H(t_N) =H(q)$
        \end{itemize}
        From which we can see
        \begin{align}
            &\sum_{t:D(t\Vert q)\le\epsilon_N} Q_{N,t} \sim 1-\exp[-N\epsilon] \\
            &\sum_{t:D(t\Vert q)\le\epsilon_N} \frac{\log S_{N,t}}N = H(q)%这个式子背后的物理是什么?
        \end{align}
        \item Schumacher's theorem, direct part: Let $\rho \in \St(\mathcal H)$, then every compression rate $R\ge S(\rho)$ is achievable
        \item Schumacher's theorem, strong converse: Let $\rho = \sum_i q_i\ket{\psi_i}\bra{\psi_i}$ be a diagonalization of $\rho$ and let $F_N$ be the fidelity of data compression for the states $\{\ket\psi_i\}$ with probabilities $\{q_i\}$, then for every $R < S(\rho)$, $\lim_{N\to \infty} F_N = 0$
        \item Entanglement dilution: Using LOCC to produce $M_N$ pairs of $\ket\Psi$ from $N$ pairs of Bell state $\ket{\Phi^+}$, the achievable rate
        \begin{equation}
            R_{\mathrm{dil}} =\liminf_{N\to\infty}\frac{M_N}N < 1/S(\Tr_A[\ket\Psi\bra\Psi])
        \end{equation}
        \item Entanglement distillation: Using LOCC to produce $M_N$ pairs of Bell state $\ket{\Phi^+}$ from $N$ pairs of $\ket\Psi$, the achievable rate
        \begin{equation}
            R_{\mathrm{dist}} = \liminf_{N\to\infty}\frac{M_N}N < S(\Tr_A[\ket\Psi\bra\Psi])
        \end{equation}
        \item Asymptotic transformations of pure entangled states:$R = S(\Tr_A[\ket\Psi\bra\Psi])/S(\Tr_A[\ket{\Psi'}\bra{\Psi'}])$
    \end{enumerate}
\end{enumerate}
% subsection quantum_data_compression (end)
% section quantum_entropy (end)
\section{Quantum algorithm} % (fold)
\label{sec:quantum_algorithm}
\subsection{Grover's quantum search algorithm} % (fold)
\label{sub:quantum_search}
\begin{enumerate}
    \item Classic model: from a function 
    \begin{equation}
        f:\{1,\cdots,N\}\mapsto\{0,1\}
    \end{equation}
    find $n$ so that $f(n) = 1$. Usually we assume that $S = |\{n|f(n)=1\}|\ll N$. Time complexity $O(N)\times O(f)$
    \item Two quantum version: 
    \begin{itemize}
        \item we have the system
        \begin{equation}
            \ket\alpha = \ket{f(1)}\ket{f(2)}\cdots\ket{f(N)}
        \end{equation}
        And define the control unitary gate
        \begin{equation}
            U = \sum_{n=1}^NZ_n\otimes\ket n\bra n 
        \end{equation}
        with Pauli gate defined by $Z_n\ket\alpha = (-)^{f(n)}\ket\alpha$. \\
        For simplicity we define the \emph{Grover's gate} $V_f$ on the control system as
        \begin{equation}
            V_f = \sum_{n=1}^N(-)^{f(n)}\ket n\bra n
        \end{equation}
        which comes from $U\ket\alpha\ket\beta = \ket\alpha(V_f\ket\beta)$\footnote{In the following we discuss the \emph{query complexity} defined by the number of uses of gate $V_f$, and ignore the elementary gates needed to perform $V_f$ (which leads to \emph{gate complexity}).}
        \item We may describe the classic search as:
        \begin{equation}
            \mathscr (\rho) = \sum_n\braket{n|\rho|n}\ket{f(n)}\bra{f(n)} \left( \Tr_A\left[U_f(\rho\otimes\ket0\bra0)U_f^\dagger\right]\right)
        \end{equation}
        And quantum version by the gate $U_f = \sum_n\ket n\bra n\otimes X^{f(n)}$, which also leads to Grover's gate $V_f$ by
        \begin{equation}
            U_f\ket\beta\ket- = (V_f\ket\beta)\ket-
        \end{equation}
        This version seems to show how quantum version of $U_f$ is more powerful than $\mathscr C_f$ and show that preparing a huge system of $\ket\alpha$ is not necessary.
    \end{itemize}
    \item The algorithm ($O(\sqrt N)$):
    \begin{enumerate}
        \item prepare system in Fourier basis state
        \begin{equation}
            \ket{e_N} = \frac 1{\sqrt N}\sum_{n=1}^N\ket n
        \end{equation}
        \item Apply Grover's gate $V_f$ \label{algo_gg}
        \item Apply the gate
        \begin{equation}
            W = 2\ket{e_N}\bra{e_N} - I
        \end{equation}\label{algo_w}
        \item Repeat steps \ref{algo_gg} and \ref{algo_w} for $k$ times, where
        \begin{equation}
            k = \frac\pi4\sqrt{\frac NS}
        \end{equation}
        \item Measure the system on computational basis, with probability of success
        \begin{equation}
            p_{\mathrm{succ}} \ge 1-\frac SN
        \end{equation}
    \end{enumerate}
    Proof of the algorithm:\\
    The input state can be expressed as:
    \begin{equation}
        \ket{e_N} = \sqrt{1-\frac SN}\ket{\psi_+} + \sqrt{\frac SN}\ket{\psi_-} := \cos\theta\ket{\psi_+} + \sin\theta\ket{\psi_-}
    \end{equation}
    where $\ket{\psi^+}$ and $\ket{\psi_-}$ are eigenstates of $V_f$ with eigenvalue $\pm 1$:
    \begin{align}
        \ket{\psi_+} &:= \frac 1{\sqrt{N-S}}\sum_{n:f(n)=0}\ket n \\
        \ket{\psi_-} &:= \frac 1{\sqrt S}\sum_{n:f(n)=1}\ket n 
    \end{align}
    And $\ket{e_N}$ and $\ket{e_N^\perp}:=-\sin\theta\ket{\psi_+} + \cos\theta\ket{\psi_-}$ are eigenstates of $W$ with eigenvalue $\pm 1$. So
    \begin{equation}
        WV_f(\cos\alpha\ket{\psi_+} + \sin\alpha\ket{\psi_-}) = \\\cos(\alpha+2\theta)\ket{\psi_+} + \sin(\alpha+2\theta)\ket{\psi_-}
    \end{equation}
    therefore
    \begin{equation}
        (WV_f)^k\ket{e_N} = \cos[(2k+1)\theta]\ket{\psi_+} + \sin[(2k+1)\theta]\ket{\psi_-}
    \end{equation}
    with $(2k+1)\theta\approx\pi/2$, we have the conclusion above.
    \item Dependence of the algorithm on the number of solutions. Quantum phase estimation algorithm allows us to find out the angle $\tau = 2\arcsin\sqrt{S/N}$ within an interval of size $1/M$ with the control-unitary gate
    \begin{equation}
        T = \sum_{m=1}^MV^m_f\otimes\ket m\bra m
    \end{equation}
    \item $O(\sqrt N)$ is the best scaling allowed by quantum mechanics\\
    \textbf{Proof}: For simplicity here we only discuss the case $S=1$, with
    \begin{equation}
        V_f = -\ket x\bra x + \sum_{n\neq x}\ket n\bra n =: V_x
    \end{equation}
    Generally the algorithm could be
    \begin{equation}
        \ket{\Psi_{k,x}} = U_k(V_x\otimes I_B)U_{k-1}\cdots U)1(V_x\otimes I_B)\ket{\Psi_0}
    \end{equation}
    with $\ket{\Psi_0}\in\mathcal H_A\otimes \mathcal H_B$ be the input system combined with a auxiliary system $B$. And we hope to get the result $\ket{\Phi_{k,x}} = \ket x\ket{\beta_{k,x}}$, which leads to the quality of the algorithm (average on $x$)
    \begin{align}
        \eta_{k}&:=\frac 1N\sum_{x=1}^N\left\Vert\ket{\Psi_{k,x}}-\ket{\Phi_{k,x}}\right\Vert^2 \\
        &\ge \frac 1N\sum_x\bigg(\left\Vert\ket{\Psi_{k,x}}-\ket{\Psi_{k}}\right\Vert - \left\Vert\ket{\Psi_{k}}-\ket{\Phi_{k,x}}\right\Vert\bigg)^2 \\
        &\ge \frac 1N\left(\sqrt{\sum_x\left\Vert\ket{\Psi_{k,x}}-\ket{\Psi_{k}}\right\Vert^2} - \sqrt{\sum_x\left\Vert\ket{\Psi_{k}}-\ket{\Phi_{k,x}}\right\Vert^2}\right)^2 \\
        &:= \left(\frac{\Vert\vec a_k\Vert - \Vert \vec b_k\Vert}{\sqrt N}\right)^2
    \end{align}
    where $\ket{\Psi_k}:=U_k U_{k-1}\cdots U_1\ket\Psi$ and the items
    \begin{align}
        \Vert\vec a_k\Vert &= \sqrt{\sum_x\left\Vert\ket{\Psi_{k,x}}-\ket{\Psi_{k}}\right\Vert^2} \le 2k \\
        \Vert\vec b_k\Vert &= \sqrt{\sum_x\left\Vert\ket{\Psi_{k}}-\ket{\Phi_{k,x}}\right\Vert^2} \\
        &\ge \sqrt{2\left(N-\sqrt N\right)}
    \end{align}
    So to promise $\eta_k\to 0$, it have to be $k = \Theta(\sqrt N)$
    \begin{itemize}
        \item It makes no difference to use $U_x = \sum_n\ket n\bra n\otimes X^{f(n)}$ because
        \begin{equation}
            U_x = I_A\otimes\ket+\bra+ + V_x\otimes\ket-\bra-
        \end{equation}
        \item The proof holds even if we use the gate $V_x^t\otimes I_B$ as one step, which means the optimal of $\Theta(\sqrt N)$ is not only the times needed to use $V_x$ but also the steps needed
    \end{itemize}
\end{enumerate}
% subsection quantum_search (end)
\subsection{Shor's algorithm} % (fold)
\label{sub:shors_algorithm}
\begin{enumerate}
    \item From period finding to factoring:
    \begin{enumerate}
        \item Take a random integer $a < N$ and check if $a$ divides $N$\label{algo_rda}
        \item If a divides $N$, you are done: this means that either $a = p$ or $a = q$. If not, then proceed to the next step
        \item Find the period of the function $f(x) = ax \mod N$. Call the period $r$
        \item If $r$ is odd, then go back to first step. If $r$ is even, then proceed to the next step
        \item Compute $x_+ = a^{r/2} + 1$ and $x = a^{r/2} - 1$.
        \item If $x_+ = 0 \mod N$, then go back to first step. Otherwise, proceed to the next step
        \item Output the solution $\{p, q\} = \{\operatorname{gcd}(N,x_+),\operatorname{gcd}(N,x_-)\}$.
    \end{enumerate}
    \item Quantum period-finding algorithm (Shor's algorithm) with the gate $U_f$
        \begin{equation}
            U_f := \sum_{x=1}^{d}\ket x\bra x\otimes S^{f(x)}
        \end{equation}
        with $f(x)$ a ``strong periodic'' function
        \begin{equation}
            f(x) = f(y) \Longleftrightarrow x = y+kr,k\in\mathbb Z
        \end{equation}
        (for example what we need in factorization $f(x) = a^x\mod N$) and shift gate $S = \sum\ket{i\oplus 1}\bra{i}$
    \begin{enumerate}
        \item Prepare system $A$ in Fourier state $\ket{e_0} = \frac 1{\sqrt{d_A}}\sum_x\ket x$ and system $B$ in $\ket0$
        \item Apply $U_f$
        \item Measure $B$ in the computational basis
        \item Measure $A$ in the Fourier basis with result $\ket{e_n}$
        \item In the easy case where $d$ is a multiple of the period, it is enough to compute the fraction $n/d$ and reduce it to minimal terms $n/d = k_0/r_0$, in this case, $r_0$ is a divisor of the period.
    \end{enumerate}
    \textbf{Proof}
    \begin{align}
        U_f\ket{e_0}\ket0 &= \frac 1{\sqrt d}\sum_{x=1}^d\ket x\ket{f(x)} \\
        &= \frac 1{\sqrt d}\sum_{x=1}^r\left(\sum_{m=0}^{M_x-1}\ket{x+mr}\right)\ket{f(x)}
    \end{align}
    After the measurement of system $B$, the state of system $A$ becomes
    \begin{equation}
        \ket{\psi_x} = \frac 1{\sqrt{M_x}}\sum_{m=0}^{M_x-1}\ket{x+mr}
    \end{equation}
    and the result of measurement of system $A$ is 
    \begin{align}
        p(n|x) &= |\braket{e_x|\psi_x}|^2 \\
        &= \frac 1{M_x}\left|\sum_{m=0}^{M_x-1}\braket{e_n|x+mr}\right|^2 \\
        &= \frac 1{M_x d}\left|\sum_{m=0}^{M_x-1}\exp\left[-\frac{2\pi\mi nmr}d\right]\right|^2
    \end{align}
    If $d = rM$ (means d is a multiple of the period), then the probability is independent on $x$
    \begin{equation}
        p_n = \frac 1r\left|\frac 1M\sum_{m=0}^{M-1}\exp\left[-\frac{2\pi\mi nm}M\right]\right|^2 = \frac 1r\sum_{k=1}^r \delta_{n,kM}
    \end{equation}
    Therefore the outcome $n = kM$, and $n/d = k/r$, so $r_0$ should be a divisor of $r$\\
    If $d$ is not a multiple of the period, it is easy to see that with large $M_x$ (or large $d$, for example $d\sim N^2$), the probability is peaked around $n = kd/r$
    \begin{itemize}
        \item The complexity:
        \begin{itemize}
            \item the gate $U_f$ with $f = a^x\mod N$: $O(L^3)$
            \item prepare the Fourier basis: the ``multiply operator'' $\ket{e_k} = M^k\ket{e_0}$
            \begin{equation}
                M = \sum_{n=1}^N\exp\left[\frac{2\pi\mi n}N\right]\ket n\bra n
            \end{equation}
            And to express in binary qubit:
            \begin{equation}
                \ket x = \ket{x_1}\ket{x_2}\cdots\ket{x_L}
            \end{equation}
            with $x_i\in\{0,1\}$ and $x = \sum_i 2^{L-i}x_i$, than we have:
            \begin{equation}
                \ket{e_0} = \ket+^{\otimes L} = (H\ket 0)^{\otimes L}
            \end{equation}
            And ``multiply operator''
            \begin{align}
                M\ket n &= \bigotimes_{i=1}^L\left[\exp\left(\frac{2\pi \mi n_i}{2^i}\right)\ket{n_i}\right] \\
                &= R_1\ket{n_1}\otimes R_2\ket{n_2}\otimes\cdots\otimes R_L\ket{n_L}
            \end{align}
            where the single-qubit gate
            \begin{equation}
                R_i := \begin{pmatrix}
                    1 & 0\\
                    0 & \exp\left(\frac{2\pi \mi}{2^i}\right)
                \end{pmatrix}
            \end{equation}
            which can be realized with $O(-\log^c\epsilon)\sim O(1)$, and so is $M^k$ (leads to multiple of $L$). And all these sums up to the complexity $O(L)$
            \item measurement on the Fourier basis: The key is to realize the Fourier gate:
            \begin{equation}
                F = \sum_{n=1}^N\ket{e_n}\bra n
            \end{equation}
            Define the control-unitary gate on the control $n$ and target $m$
            \begin{equation}
                C_i^{(mn)} := I^{(m)}\otimes\ket0_n\bra0_n + R^{(m)}_i\otimes\ket1_n\bra1_n
            \end{equation}
            and gates (the Hadamard gate $H$ defined above)
            \begin{align}
                U_1 &:= C_L^{(1L)}\cdots C_3^{(13)}C_2^{(12)}H^{(1)} \\
                U_2 &:= C_{L-1}^{(2L)}\cdots C_3^{(24)}C_2^{(23)}H^{(2)} \\
                &\vdots \\
                U_L &:= H^{(L)}
            \end{align}
            And the Fourier gate
            \begin{equation}
                F = U_L U_{L-1}\cdots U_2 U_1
            \end{equation}
            with complexity $O(L^2)$ to realizing a physical Fourier transform\footnote{Classically we need $\Theta(N\log N)$ to perform \emph{fast Fourier transform}. It does not mean we could do better in quantum computation because we cannot get all factors in $F\ket k$.}
        \end{itemize}
        All above sums up to $O(L^3)$
    \end{itemize}
    %RSA code?
\end{enumerate}
% subsection shors_algorithm (end)
% section quantum_algorithm (end)
\end{document}