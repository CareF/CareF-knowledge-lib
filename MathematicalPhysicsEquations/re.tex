\documentclass[12pt,a4paper]{article}
\usepackage{ctex}%ctex套用英文标题格式(建议在英文论文混排中文时使用),ctexcap套用中文格式(等同于\documentclass{ctexart})
\usepackage[a4paper,top=0.75in,bottom=1in,left=1in,right=1in]{geometry}

\usepackage{amsmath,amssymb,array,esint}%数学公式类宏包;最末为积分符号拓展
\allowdisplaybreaks[2]%允许多行公式间换页,用//*表示不允许换页
\newcommand\dif{\mathrm{d}}
\newcommand\diff{\,\mathrm{d}}
\usepackage{bm}%加粗(用于vector)
\renewcommand*{\vec}[1]{\bm{#1}}%矢量的格式,这里是加粗

\usepackage[CJKbookmarks]{hyperref}

\usepackage{multirow,longtable}
%以下实现带圈的脚注
\usepackage{pifont}
\usepackage[perpage,symbol*]{footmisc}
\DefineFNsymbols{circled}{{\ding{192}}{\ding{193}}{\ding{194}}
{\ding{195}}{\ding{196}}{\ding{197}}{\ding{198}}{\ding{199}}{\ding{200}}{\ding{201}}}
\setfnsymbol{circled}

\usepackage{cases}


\renewcommand{\[}{\ $\displaystyle}
\renewcommand{\]}{$\ }%$
\newcommand{\fdif}[2]{\ensuremath{\frac{\dif #1}{\dif #2}}}
\newcommand{\fdifsq}[2]{\ensuremath{\frac{\dif^2 #1}{\dif #2^2}}}
\newcommand{\pard}[2]{\ensuremath{\frac{\partial #1}{\partial #2}}}
\newcommand{\pardsq}[2]{\ensuremath{\frac{\partial^2 #1}{\partial #2^2}}}
\newcommand\mi{\mathrm{i}}
\newcommand\e{\mathrm{e}}
\newcommand\res{\mathop{\rm res~}}
\newcommand{\summ}[2][n]{\sum_{#1=#2}^\infty}
\usepackage{mathrsfs}
\begin{document}
\title{数理方程复习整理}
\author{吕铭\quad 物理21}
\maketitle
%\renewcommand\theenumi{\chinese{enumi}}
%\renewcommand\theenumii{\arabic{enumii}}
%\renewcommand\labelenumii{\theenumii.}
\section{二阶线性常微分方程}
  \subsection{级数解的一般原理}
	\begin{enumerate}
	 \item 一般的,对于常微分方程初值问题
	 	\begin{eqnarray*}
	 		\fdifsq{w}{z}+p(z)\fdif{w}{z}+q(z)w = 0 \\
	 		w(z_0) = c_0,\quad w'(z_0) = c_1
	 	\end{eqnarray*}
	 	如果\[p(z)\]和\[q(z)\]在圆\[|z-z_0|<R\]内单值解析,则有唯一的一个解\[w(z) = c_0 w_1(z)+c_1 w_2(z)\],且\[w(z)\]在这个圆内单值解析。而后可以通过解析延拓拓展到圆外。
	 	\item 若\[z_0\]是方程的奇点,在\[0<|z-z_0|<R\]内,方程的两个线性无关解是
	 	\begin{eqnarray*}
	 	  w_1(z) &=& (z-z_0)^{\rho_1}\summ{-\infty}c_k(z-z_0)^k \\
	 	  w_2(z) &=& gw_1\ln(z-z_0) + (z-z_0)^{\rho_2}\summ{-\infty}d_k(z-z_0)^k
	 	\end{eqnarray*}
	 	其中\[\rho_1\]、\[\rho_2\]和\[g\]都是常数。
	 	\item 当\[z_0\]是\[p(z)\]的不超过一阶极点,\[q(z)\]的不超过二阶极点时,称\[z_0\]为方程的正则奇点,此时级数只含有有限个负幂项,可以写作:
	 	\begin{eqnarray*}
	 	  w_1(z) &=& (z-z_0)^{\rho_1}\summ{0}c_k(z-z_0)^k ,\quad c_0\neq 0 \\
	 	  w_2(z) &=& gw_1\ln(z-z_0) + (z-z_0)^{\rho_2}\summ{0}d_k(z-z_0)^k, \quad g\neq0\mbox{或}d_0\neq 0
	 	\end{eqnarray*}
	 	称之为正则解。\[g=0\]时,两个解的形式相同。\[\rho_1\]和\[\rho_2\]称为正则解的指标。
	 \item 对于\[\mathrm{Re} \rho_1\geq \mathrm{Re} \rho_2\],则:
	 	\begin{tabbing}
	 	当\[\rho_1 - \rho_2 \neq\]非负整数时 \quad \= 第二解一定不含对数项\\
	 	当\[\rho_1 - \rho_2 =0\]时 \> 第二解一定含对数项\\
	 	当\[\rho_1 - \rho_2 =\]正整数时 \> 第二解可能不含对数项
	 	\end{tabbing}
	\end{enumerate}
  \subsection{指标方程}
	 	\begin{eqnarray*}
	 		&\rho(\rho -1) + p_0\rho + q_0 = 0 \\
	 		&p_0 = \lim\limits_{z\to z_0}(z-z_0)p(z),\quad q_0 = \lim\limits_{z\to z_0} (z-z_0)^2q(z)
	 	\end{eqnarray*}
  \subsection{朗斯基行列式(Wronskian)}
	 	$$
	 	  \Delta(z) = \begin{vmatrix}w_1(z) & w_2(z) \\ w'_1(z) & w'_2(z)\end{vmatrix} = w_1\fdif{w_2}{z} - w_2\fdif{w_1}{z}
	 	$$
	 	原方程交叉相乘相减可以得到:
	 	\begin{eqnarray*}
	 	  &&\fdif{\Delta(z)}{z} + p(z)\Delta(z) = 0 \\
	 	  \Rightarrow &&\Delta(z) = \Delta(z_0)\e^{-\int_{z_0}^{z} p(z)\diff z}
	 	\end{eqnarray*}
  \subsection{主项平衡法}
    对于非正则奇点,解的形式为:
	   $$
	     w(z) = A\e^{S(z)}z^\rho\summ{0}c_nz^n,\quad c_0 = 1
	   $$
	   其中\[A\e^{S(z)}z^\rho\]称为方程解的主项。每次仅保留方程的主要部分,分次求的\[S(z) = S_0(z) + S_1(z)+\cdots +S_{n-1}(z)\]的方法称为主项平衡法。\\
	   对于方程的非正则奇点,通过主项平衡法求解得到的级数不一定是收敛的,它往往只是渐近级数。
  \subsection{常微分方程的积分解法}
    对于二阶线性齐次常微分方程
	 $$
	 	p_0(z)\fdifsq{u}{z} + p_1(z)\fdif{u}{z} + p_2(z)u = 0
	 $$
	 定义算符\[L = p_0(z)\fdifsq{}{z} + p_1(z)\fdif{}{z} + p_2(z)\],引入积分变换
	 $$
	 	u(z) = \int_C K(z,t)v(t)\diff t
	 $$
	 转而求\[v(t)\]。其中\[K(z,t)\]称为积分变换的核,路径\[C\]要根据具体的微分方程选取。
	 
	 若找到合适的线性微分算子\[M_t\]使得:
	 $$
	 	L_z[K(z,t)] = M_t[K(z,t)]
	 $$
	 再利用分部积分,使得
	 $$
	   L[u] = \int_C K(z,t)\overline M_t[v(t)]\diff t + \{Q[K,v]\}_C
	 $$
	 其中\[\overline M_t\]称为\[M_t\]的伴随算子,\[Q\]是作用在\[K\]和\[v\]上的双线性伴式,\[\{~\}_C\]表示沿着路径\[C\]的变化。
	 
	 当\[\overline M_t[v(t)] = 0\]且\[\{Q[K,v]\}_C = 0\]时,微分方程的解为:
	 $$
	   u(z) = \int _C K(z,t)v(t)\diff t
	 $$
	 
	 常见的积分变换包括:
	 \begin{enumerate}
	  \item 拉普拉变换(Laplace transform):\[K(z,t) = \e^{zt}\]用于拉普拉斯型方程:
	  $$
	   (a_0 z + b_0)\fdifsq{u}{z} + (a_1 z +b_1)\fdif{u}{z} + (a_2 z +b_2) = 0
	  $$
	  \item 欧拉变换(Euler's transform):\[K(z,t) = (z-t)^\mu\]用于Fox型方程:
	  $$
	   (a_0 z^2 + b_0 z + c_0)\fdifsq{u}{z} + (b_1 z +c_1)\fdif{u}{z} + c_2u=0
	  $$
	  \item 梅林变换(Mellin transform):\[K(z,t) = z^t\]
	 \end{enumerate}
  \subsection{一些方程实例}
    \subsubsection{欧拉方程(Euler equation)}
	 $$
	   z^2\fdifsq{w}{z} + pz\fdif{w}{z} + qw = 0
	 $$
	 解的形式是\[w = z^\rho\],当\[\rho_1 = \rho_2\]时,\[w_2 = z^\rho \ln z\]。
	 \subsubsection{超几何方程(Hypergeometric equation)}
	 	$$
	 	  z(1-z)\fdifsq{w}{z} + [\gamma - (1+ \alpha + \beta)z]\fdif{w}{z} -\alpha\beta w = 0
	 	$$
	 	奇点\[z=1\],\[z=1\],\[z=\infty\]。
	 	
	 	超几何方程是~Fox~型方程,应用欧拉变换(\[\mu =-\alpha\]或\[\mu = -\beta\])得到积分解(\[\mu = -\alpha\],\[\mathrm{Re}(\gamma)>\mathrm{Re}(\beta)>0\]时):
	 	\begin{eqnarray*}
	 	 w(z) &=& A\int_1^\infty t^{\alpha - \gamma}(1-t)^{\gamma - \beta - 1}(z-t)^{-\alpha}\diff t\\
	 	 &=&A'\int_0^1 t^{\beta -1}(1-t)^{\gamma - \beta -1}(1-zt)^{-\alpha}\diff t
	 	\end{eqnarray*}
	 \subsubsection{勒让德方程(Legendre equation)}
	 \newcommand{\fp}[2][l]{\ensuremath{\mathrm{P}_{#1}\left(#2\right)}}
	 	$$
	 		(1-x^2)\fdifsq{y}{x} - 2x\fdif{y}{x} + l(l+1)y =\fdif{}{x}\left[(1-x^2)\fdif{y}{x}\right]+l(l+1)y = 0
	 	$$
	 	奇点\[x=\pm 1\]和\[x=\infty\](都是正则奇点)。
	 	\begin{enumerate}
		\item 在\[x=0\]的邻域内的级数解:
	 	\begin{eqnarray*}
	 		y(x) &=& c_0y_1(x) + c_1 y_2(x) \\
	 		y_1(x) &=& \summ{0} \frac{2^{2n}\Gamma\left(n-\frac l2\right)\Gamma\left(n+\frac{l+1}2\right)}{(2n)!\Gamma\left(-\frac l2\right)\Gamma\left(\frac{l+1}2\right)}x^{2n}\\
	 		y_2(x) &=& \summ{0}\frac{2^{2n}\Gamma\left(n-\frac{l-1}2\right)\Gamma\left(n+1+\frac l2\right)}{(2n+2)!\Gamma\left(-\frac{l-1}2\right)\Gamma\left(1+\frac l2\right)}x^{2n+1}
	 	\end{eqnarray*}
	 	在奇点\[z = \pm 1\]处的行为:
	 	\begin{eqnarray*}
	 		y_1(x) &\sim & \mbox{常数}\times \ln \frac{1}{1-x^2} \\
	 		y_2(x) &\sim & \mbox{常数}\times \ln \frac{1+x}{1-x}
	 	\end{eqnarray*}
	 	因此如果对于\[y_1\]和\[y_2\]做解析延拓,得到的一定是多值函数。
	 	\item 在\[x=1\]的领域内的正则集:
	 	\begin{eqnarray*}
	 	 \fp{x} &=& \summ{0} \frac 1{(n!)^2}\frac{\Gamma(l+n+1)}{\Gamma(l-n+1)}\left(\frac{x-1}{2}\right)^n \\
	 	 \mathrm Q_l(x) &=& \frac 12 \fp{x}\left[ \ln\frac{x+1}{x-1} - 2\gamma - 2\psi (l+1)\right] \\
	 	 	&&\quad +\summ{0}\frac{1}{(n!)^2}\frac{\Gamma(l+n+1)}{\Gamma(l-n+1)}(\psi(n+1)+\gamma)\left(\frac{x-1}2\right)^n
	 	\end{eqnarray*}
	 	\end{enumerate}
	 	
	 	勒让德方程是~Fox~型方程,应用欧拉变换(\[\mu = -l\]),对于\[l = n\]为非负整数的清晰,可以得到勒让德多项式的积分形式(Schl\"ofli~公式)
	 	$$
	 	  \fp[n]{x} = \frac{1}{2\pi\mi}\int_C\frac{(t^2-1)^2}{2^n(t-x)^{n+1}}\diff t
	 	$$
	 	根据留数定理可以得到~Rodrigues~公式:
	 	$$
	 	  \fp[n]{x} = \frac 1{2^n n!}\frac{\dif^n}{\dif x^n}(x^2 - 1)^n
	 	$$
	 	取围道\[|t-x| = \sqrt{|x^2 -1|}\]可以得到~Laplace~第一积分表示:
	 	$$
	 	  \fp[n]{x} = \frac{1}{\pi}\int_0^\pi\left(x+\sqrt{x^2 - 1}\cos \varphi\right)^n\diff\varphi
	 	$$
	 	
	 \subsubsection{连带勒让德方程}
	  $$
	    \fdif{}{x}\left[(1-x^2)\fdif{y}{c}\right]+\left[l(l+1)-\frac{m^2}{1-x^2}\right]y = 0
	  $$
	  解为:
	  \begin{eqnarray*}
	    y &=& c_1y_1(x) + c_2y_2(z)\\
	    y_1(z) &=& (1-z^2)^{m/2}\mathrm{P}_{l}^{(m)}(z) \\
	    y_2(z) &=& (1-z^2)^{m/2}\mathrm{Q}_{l}^{(m)}(z)
	  \end{eqnarray*}
	  \[y_1(z)\]在\[z=1\]点有界,在\[z=-1\]点发散;\[y_2(z)\]在\[z=\pm 1\]均发散。
	  
	 \subsubsection{超球微分方程}
	 	$$
	 	  (1-z^2)v'' - 2(m+1)zv' + [\lambda - m(m+1)]v = 0
	 	$$
	 \subsubsection{贝塞尔方程(Bessel equation)}
	 \newcommand{\fj}[2][\nu]{\ensuremath{\mathrm{J}_{#1}\left(#2\right)}}
	 \newcommand{\fn}[2][\nu]{\ensuremath{\mathrm{N}_{#1}\left(#2\right)}}
	 	$$
	 	  \fdifsq{w}{z} + \frac 1z\fdif{w}{z} + \left(1-\frac{\nu^2}{z^2}\right)w = 0\quad(\mathrm{Re}\nu \geq 0)
	 	$$
	 	\[\nu\]阶贝塞尔方程,\[z=0\]是正则奇点,\[z=\infty\]是非正则奇点。在\[z=0\]的邻域内的解:
	 	\begin{eqnarray*}
	 	  w &=& c_0 \fj{z} + c_1\fj[-\nu]{z} ,\quad (\nu \notin N) \\
	 	    &\mbox{或}& c_0 \fj{z} + c_1\fn{z} \\
	 	  \fj{z} &=& \summ{0} \frac{(-)^n}{n!\Gamma(n+\nu+1)}\left(\frac z2\right)^{2n+\nu}\\
	 	  \fn{z} &=& \frac{\cos \pi\nu \fj{z} - \fj[-\nu]{z}}{\sin \pi\nu} 
	 	\end{eqnarray*}
	 	其中\[\fj{z}\]和\[\fn{z}\]分别称为\[\nu\]阶贝塞尔函数(Bessel functions)和诺依曼函数(Neumann function,又称第二类贝塞尔函数)。
	 	
	 	无穷远处的解
	 	$$
	 	  w(z) = A\frac{\e^{\pm\mi z}}{\sqrt{z}}\summ{0}\frac{c+k}{z^k}
	 	$$
	 	
	 	积分解:做变换\[w = z^\nu u(z)\]变为拉普拉斯型方程,应用积分变换得到:
	 	\begin{eqnarray*}
	 	  \fj{z} &=& \frac{(z/2)^\nu}{\sqrt{\pi}\Gamma(\nu + \frac{1}{2})}\int_{-1}^{1}\e^{\mi zt}(1-t^2)^{\nu-\frac 12}\diff t \\
	 	  &=& \frac{(z/2)^\nu}{\sqrt{\pi}\Gamma(\nu + \frac{1}{2})} \int_0^\pi\e^{\mi z\cos \theta}\sin ^{2\nu}\theta \diff \theta \quad\mbox{(Poisson积分表达式)}
	 	\end{eqnarray*}
	 
	 \subsubsection{球贝塞尔方程(Spherical Bessel equation)}
	  $$
	    \frac 1{z^2}\fdif{}{z}\left(z^2\fdif{y}{z}\right) + \left[ 1-\frac{l(l+1)}{z^2}\right]y = 0
	  $$
	  \[z=0\]是正则奇点,\[z=\infty\]是非正则奇点。解:
	  \begin{eqnarray*}
	   y &=& c_0\mathrm j_l(z) + c_1\mathrm n_l(z) \\
	   \mathrm j_l(z) &=& \sqrt{\frac \pi{2z}}\fj[l+1/2]{z}\\
	     &=& \frac{\sqrt \pi}{2}\summ{0} \frac{(-)^n}{n!\Gamma(n+l+3/2)}\left(\frac z2\right)^{2n+l} \\
	   \mathrm n_l(z) &=& \sqrt{\frac \pi{2z}}\fn[l+1/2]{z} \\
	     &=& (-)^{l+1}\frac{\sqrt \pi}{2}\summ{0} \frac{(-)^n}{n!\Gamma(n-l+1/2)}\left(z2\right)^{2n-l-1}
	  \end{eqnarray*}

\section{建立模型方程}
	\begin{enumerate}
	 \item 波动方程(双曲型方程):小振动近似和胡克定律下,可以得到
	 $$
	   \pardsq{u}{t} - a^2\nabla^2u = f
	 $$
	 \item 热传递方程和扩散方程(抛物型方程):似稳近似,介质各向同性
	 $$
	   \pard{u}{t} - \kappa \nabla^2 u = f
	 $$
	 \item 稳恒状态(椭圆型方程):
	 \begin{enumerate}
	  \item 泊松方程(Poisson's equation):\[\nabla^2u = -f\]
	  \item 拉普拉斯方程(Laplace's equation):\[\nabla^2u = 0\]
	  \item 亥姆霍兹方程(Helmholtz equation):\[\nabla^2u + k^2u = 0\]
	 \end{enumerate}
	\end{enumerate}
    \subsection{边界条件}
     \begin{enumerate}
      \item 第一类边界条件:\[u|_\Sigma = \phi(\Sigma,t)\]
      \item 第二类边界条件:\[\left.\pard{u}{\vec{n}}\right|_\Sigma = f(\Sigma,t)\]
      \item 第三类边界条件:\[\left[\pard{u}{\vec n} + hu\right]_\Sigma = f(\Sigma,t)\]
     \end{enumerate}
     另外对于多种介质还有连接条件。
  \subsection{方程适定性}
  考察解的存在性、唯一性和稳定性。通常可以用守恒量,如能量来检验。
  
\section{线性偏微分方程的通解}
  记\[D_i^n \equiv \frac{\partial^n}{\partial i^n}\]
   \subsection{常系数线性齐次偏微分方程}
     $$
       L(D_x,D_y)u \equiv \left(\sum_{0\le i,j\le n} A_{ij} D_x^iD_y^j\right)u = 0
     $$
     分解为\[L(D_x,D_y) = A_{n0}\prod_k(D_x - \alpha_k D_y - \beta_k)^{n_k}\],相应的通解为:
     $$
       u = \sum_k \left[\sum_{i=0}^{n_k-1}x^i\e^{\beta_k x}\phi_{ik}(y+\alpha_k)\right]
     $$
    \subsubsection{波动方程的行波解}
    波动方程的初值问题:
    \begin{eqnarray*}
     \pardsq{u}{t} - a^2\pardsq{u}{x} =& 0 &-\infty < x < \infty ,t>0\\
     \left. u(x,t)\right|_{t=0} = &\phi(x) &-\infty < x < \infty \\
     \left. \pard{u}{t} \right|_{t=0} = &\psi(x) &-\infty < x < \infty
    \end{eqnarray*}
     的行波解,又称达朗贝尔解(d'Alembert's formula):
     $$
       u(x,t) = \frac 12[\phi(x-at)+\phi(x-at)] + \frac 1{2a}\int_{x-at}^{x+at} \psi(\xi)\diff \xi
     $$
   \subsection{常系数线性非齐次方程}
    $$
      L(D_x,D_y)u = f(x,y)
    $$
    形式地\[u_0 = \frac{1}{L(D_x,D_y)}f(x,y)\]
    \begin{enumerate}
     \item \[f(x,y) = \e^{ax+by}\]
     	$$
     	  \frac{1}{L(D_x,D_y)}\e^{ax+by} = \frac{1}{L(a,b)}\e^{ax+by}
     	$$
     	例外的\[L(a,b) = 0\]的情况:
     	$$
     	  \frac{1}{bD_x - aD_y}e^{ax+by} = \frac{x}{b}e^{ax+by}
     	$$
     \item \[f(x,y) = e^{ax+by}g(x,y)\]
     	$$
     	  \frac{1}{L(D_x,D_y)}\e^{ax+by} = \e^{ax+by}\frac{1}{L(D_x + a,D_y + b)}
     	$$
     \item \[f(x,y) = x^my^n\]\\
     	将\[\frac{1}{L(D_x,D_y)}\]展开成幂级数形式求解。
     \item \[f(x,y) = g^{(n)}(ax+by)\]且\[L(D_x,D_y)\]是\[n\]次齐次的
     	$$
     	  \frac{1}{L(D_x,D_y)}g^{(n)}(ax+by) = \frac{1}{L(a,b)}g(ax+by)
     	$$
     	例外的\[L(a,b) = 0\]的情况:
     	$$
     	  \frac{1}{(D_x - \alpha D_y)^n}f^{(n)}(x)\psi(y + \alpha x) = f(x)\psi(y + \alpha x)
     	$$
     \end{enumerate}
     \subsection{特殊的变系数线性齐次偏微分方程}
      对于\[L(D_x, D_y)\]由形如\[x^my^nD_x^mD_y^n\]组成的,可以通过变换:
      $$
        x = \e^t,\qquad y = \e^s
      $$
      变为常系数微分方程。
 
\section{内积空间与函数空间}
	\subsection{内积空间}
	\begin{enumerate}
	 \item 矢量空间
	 \item 内积空间
	 \item 欧几里德空间(Euclidean space):有内积的实矢量空间
	 \item 酉空间(unitary space):具有内积的复矢量空间
	 \item 希尔伯特空间(Hilbert space):完备的内积空间,也即~Cauchy~序列的极限仍保持在该空间内的内积空间
	 \item \[\{\hat x_i\}\]是一组完备基的充要条件:
	 	\begin{enumerate}
	 	 \item \[\vec x=0 \Leftrightarrow \forall i.(\hat x_i,\vec x) = 0\]
	 	 \item \[\forall \vec x.\vec x= \sum_i (\hat x_i,\vec x)\hat x_i\]
	 	 \item 贝塞尔不等式(Bessel's inequality)取等号:\[\forall \vec x. ||\vec x||^2 = \sum_i |(\hat x_i,\vec x)|^2\]
	 	 \item 帕塞瓦尔定理(Parseval's theorem)成立:\[\forall \vec x,\vec y. (\vec y,\vec x) = \sum_i (\vec y,\hat x_i)(\hat x_i,\vec x)\]
	 	\end{enumerate}
	 \end{enumerate}
	 
	 \subsection{函数空间}
	 \begin{enumerate}
	 \item 函数空间:元素是定义在一定区间(可以是有界区间、无界区间或半无界区间,为确定起见,以下不妨写为\[a\le x \le b\])上的复值平方可积函数。
	 \item \[f(x)\]是零函数\[\Leftrightarrow (f,f) = 0\]
	 \item 逐点收敛:展开式应该对区间\[[a, b]\]内的每一点\[x\]都成立
	 \item 平均收敛:展开式左右两端可以相差一个广义的零函数
	 \item 广义函数:给定一组正交归一完备集\[\{fi(x)\],任意具有良好收敛性质的系数序列\[ci\]都定义了唯一的函数
	 	$$
	 	  f(x) = \summ{1} c_nf_n(x)
	 	$$
	 \end{enumerate}
	 \subsection[delta~函数]{\[\delta\]函数}
	 (形式的,后取极限)
	 	\begin{itemize}
	 	  \item \[\int_{-\infty}^{\infty} f(x)\delta(x)\diff x = f(0)\](定义)
	 	  \item \[\int_{-\infty}^{\infty}f(x)\delta'(-x)\diff x = -f'(0)\]
	 	  \item \[\delta(g(x)) = \left.\frac{\delta(x-x_0)}{|g'(x_0)|}\right|_{g(x_0) = 0}\]
	 	  \item \[\delta(x) = \frac{1}{2\pi}\int_{-\infty}^\infty \e^{\mi kx}\diff x\](形式的,后做积分)
	 	\end{itemize}
	
\section{分离变量方法}
	\subsection{一般理论}
		\begin{enumerate}
		 \item 伴算符:\[\forall u,v.(v,\bm{L}u) = (\bm{M}v,u)\Leftrightarrow\]\[\bm M\]与\[\bm L\]互为伴算符。伴算符都是互伴的。
		 \item 自伴算符:伴算符是自身的算符。
		 \item 允许函数类:
		 	\begin{enumerate}
		 	  \item 定义在给定区间上
		 	  \item 具有一定的连续性(属于某一个~Hilbert~空间)
		 	  \item 满足一定的边界条件(通常是一二三类边界条件或者周期性边界条件)
		 	\end{enumerate}
		 	*不能脱离边界条件的约束来讨论算符的互伴和自伴性。
		 \item 自伴算符的本征值问题:方程\[\bm{L}y(x) = \lambda y(x)\](边界条件定义在\[\bm L\]中)
		 	\begin{enumerate}
		 	 \item 自伴算符的本征值是实数
		 	 \item 自伴算符不同本征值的本征函数相互正交
		 	 \item 自伴算符的本征值问题与泛函
		 	 	$$
		 	 	  \lambda[y] = \frac{(y^*(x),\bm{L}y(x))}{(y^*(x),y(x))}
		 	 	$$
		 	 	的变分极值问题\[\delta \lambda[y] = 0\]等价。
		 	 \item 如果自伴算符的本征值可数,最小本征值有限,并且\[\lim_{n\to\infty}\lambda_n \to +\infty\]则其本征函数的序列是完备的
		 	 \item 可以通过变分法在有限函数集中获得本征值问题的逼近。这样逼近获得的本征值总是比真实值偏大。
		 	\end{enumerate}
		\end{enumerate}
	\subsection{斯图姆-刘维尔型方程(Sturm–Liouville equation)}
		$$
		  \fdif{}{x}\left[p(x)\fdif{y}{x}\right] + [\lambda \rho(x) - q(x)]y = 0
		$$
		\begin{enumerate}
		 \item 定义算符\[\bm{L} = \frac{1}{\rho(x)}\left\{-\fdif{}{x}\left[p(x)\fdif{}{x}\right] + q(x)\right\}\]
		 \item 定义内积\[(y_1,y_2) \equiv \int_a^b y_1^*(x)y_2(x)\rho(x)\diff x\](要求\[\rho(x)\ge 0\]且不恒为零)
		 \item 边界条件(使得\[\bm L\]是自伴的):当\[p(x),q(x)\in \mathbb{R} (x)\]时
		 $$
		 (y_1,\bm{L}y_2) - (\bm{L}y_1,y_2) = \left. p(x)\left(y_1^*\fdif{y_2}{x} - \fdif{y_1^*}{x}y_2\right)\right|_{a_1}^{a_2} = 0
		 $$
		 	\begin{enumerate}
		 	 \item 非周期性条件:\[\left.p(x)\left(y_1^*\fdif{y_2}{x} - \fdif{y_1^*}{x}y_2\right)\right|_{x= a_i}= 0\]
		 	 \begin{itemize}
		 		\item \[p(x)|_{x = a_i}\neq 0\],第一二三类边界条件等价于\[\left[y_1^*\fdif{y_2}{x} - \fdif{y_1^*}{x}y_2\right]_{x=a_i} = 0\]
		 		\item \[p(x)|_{x = a_i} = 0\],如果\[a_i\]点是方程正则奇点,边界条件通常是有界条件
		 		\item \[p(x)|_{x = a_i} = 0\],且\[a_i\]点是方程非正则奇点,可通过要求函数平方可积(即“归一化条件”)而确定本征值
		 	 \end{itemize}
		 	 \item 周期性条件:\[[p(x),q(x),\rho(x),y(x),y'(x)]_{a_1}^{a_2} = 0\]
		 	\end{enumerate}
		\end{enumerate}
		
		这样的定义下,S-L方程的本征值问题满足上面的一般性讨论。
		\subsubsection{实系数~S-L~方程的简并}
		\begin{enumerate}
		 \item 若本征函数是复的,且实部和虚部线性无关,则此本征值问题二重简并(实部和虚部都是本征函数)
		 \item 满足上述非周期性边界条件的两个线性无关的实本征函数对应不同的本征值
		\end{enumerate}
		
		\subsubsection{实系数~S-L~方程的其他细节}
		\begin{enumerate}
		 \item 第一类和第二类边界条件下,当\[p(x),q(x),\rho(x)\ge 0\]时,本征值有下界,上界\[\infty\],本征函数是完备的。
		 \item 第一类边界条件下,本征值有下界时,令\[\lambda_1<\lambda_2<\cdots\],则\[\lambda_n\]对应的本征函数在端点以外有\[n-1\]个零点(也称节点),特别的,最小的本征值对应的本征函数没有节点
		 \item 对于分段+连接条件的情况,可能会需要一个分段的权重因子。因子在每段内可以是常数。%重要!
		\end{enumerate}
		
	\subsection{分离变量方法的要求}
	\begin{enumerate}
	 \item 定解问题是线性的(偏微分方程、边界条件、初始条件)
	 \item 定解问题有齐次的边界条件
	 \item 本征函数构成完备正交基
	 \item 级数具有较好的收敛性
	\end{enumerate}
	
\section{本征值问题与特殊函数}	
	\subsection{拉普拉斯算符(Laplace operator)}
		\begin{longtable}[c]{l>{\rule{0em}{2em}$\displaystyle\nabla^2 \equiv} l<{$}}
		坐标系	&\multicolumn{1}{l}{坐标表示} \\
		二维直角坐标系	&\pardsq{}{x} + \pardsq{}{y} \\
		二维极坐标系	&\frac 1r\pard{}{r}\left(r\pard{}{r}\right) + \frac{1}{r^2}\pardsq{}{\phi}\\
		三维直角坐标系	&\pardsq{}{x} + \pardsq{}{y} + \pardsq{}{z}\\
		三维柱坐标系	&\frac 1r\pard{}{r}\left(r\pard{}{r}\right) + \frac{1}{r^2}\pardsq{}{\phi} + \pardsq{}{z}\\
		三维球坐标系	&\frac{1}{r^2}\pard{}{r}\left(r^2\pard{}{r}\right) + \frac{1}{r^2\sin\theta}\pard{}{\theta}\left(\sin\theta\pard{}{\theta}\right) + \frac{1}{r^2\sin^2\theta}\pardsq{}{\phi}
		\end{longtable}
		
	\subsection{亥姆霍兹方程(Helmholtz equation)的分离变量}
		\begin{enumerate}
		  \item 柱坐标系
		  	\begin{eqnarray*}
		  	  &&\frac 1r\pard{}{r}\left(r\pard{u}{r}\right) + \frac{1}{r^2}\pardsq{u}{\phi} + \pardsq{u}{z} + ku = 0\\
		  	  &\Rightarrow &\left\{\begin{array}{>{\rule{0em}{2em}\displaystyle}ll}
		  	  \fdifsq{Z}{z} + \lambda^2 Z = 0 \\
		  	  \fdifsq{\Phi}{\phi} + \nu^2 \Phi = 0 \quad \mbox{($2\pi$周期条件)}\\
		  	  \frac 1r\fdif{}{r}\left(r\fdif{R}{r}\right) + \left( k^2 - \lambda^2 - \frac{\nu^2}{r^2}\right) R = 0 & \cdots\cdots \mbox{贝塞尔方程}
		  	  \end{array}\right.
		  	\end{eqnarray*}
		  \item 球坐标系
		    \begin{eqnarray*}
		      &&\frac{1}{r^2}\pard{}{r}\left(r^2\pard{u}{r}\right) + \frac{1}{r^2\sin\theta}\pard{}{\theta}\left(\sin\theta\pard{u}{\theta}\right) + \frac{1}{r^2\sin^2\theta}\pardsq{u}{\phi} + k^2 u = 0\\
		      &\Rightarrow &\left\{\begin{array}{>{\rule{0em}{2em}\displaystyle}ll}
		  	  \fdifsq{\Phi}{\phi} + m^2 \Phi = 0 \quad \mbox{($2\pi$周期条件)}\\
		  	  \frac 1{r^2}\fdif{}{r}\left(r^2\fdif{R}{r}\right) + \left( k^2 - \frac{l(l+1)}{r^2}\right) R = 0 & \cdots\cdots\mbox{球贝塞尔方程}\\
		  	  \frac 1{\sin \theta}\fdif{}{\theta}\left(\sin \theta \fdif{\Theta}{\theta}\right) + \left(l(l+1) - \frac{m^2}{\sin ^2\theta}\right)\Theta = 0 & \cdots\cdots\mbox{连带勒让德方程}
		  	  \end{array}\right.
		    \end{eqnarray*}
		\end{enumerate}
			
	\subsection{常用本征函数表}
	对于\[\bm L u + \lambda u = 0\]\footnote{此处对于本征值定义和上面的相差一个符号},其中\[\bm{L} = \frac{1}{\rho(x)}\fdif{}{x}\left[p(x)\fdif{}{x}\right] + q(x)\]可以确定权重。
	\newcommand{\border}[3]{\ensuremath{\left. #1 \right|_{#2 = #3}}}
		\subsubsection{非特殊函数}
		\begin{longtable}[c]{c|*{4}{>{\rule[-0.5em]{0em}{2.5em}$\displaystyle}c<{$}}}
			$\bm L $	&\multicolumn{1}{c}{边界条件}	&\multicolumn{1}{c}{本征函数}	&\multicolumn{1}{c}{本征值}	&\multicolumn{1}{c}{归一化系数\footnote{权重因子默认为1,后同。}} \\\hline\endhead
			\multirow{4}{*}{\rule{0em}{5em}$\displaystyle \fdifsq{}{x}$}
				&\border{u}{x}{0,l}=0 	&\sin\left(\frac{n\pi}l x\right)\footnote{\[n = 1,2,3,\cdots\],后同。}	&\left(\frac{n\pi}l\right)^2	&\sqrt{\frac 2l}\\*
				&\border{\fdif{u}{x}}{x}{0,l} = 0	&1;\cos\left(\frac{n\pi}l x\right)	&0;\left(\frac{n\pi}l\right)^2	&\sqrt{\frac {1}l};\sqrt{\frac {2}l}\\*
				&\mbox{第三类}	&\sin\left(\lambda_n x +\phi\right)	&-\lambda_n^2\footnote{\[\lambda_n\]是边界条件相关的超越方程的根。}	&\mbox{略}	\\*
				&\mbox{周期性}	&1;\e^{\pm \mi\frac{2\pi}{T}nx}	&0;\left(\frac{2n\pi}T\right)^2	&\sqrt{\frac 1T} \\\hline
			$\displaystyle r\fdif{}{r}\left(r\fdif{}{r}\right)$	&u|_{r = a,b} = 0	&\sin\left(n\pi \frac{\ln (r/a)}{\ln (b/a)}\right)	&\left(\frac{n\pi}{\ln (b/a)}\right)^2	&\left(\frac{1}{2}\ln \frac ba\right)^{-\frac 12} \\			
			$\displaystyle \fdif{}{r}\left(r^2\fdif{}{r}\right)$	&u|_{r = a,b} = 0	&\sqrt{\frac 1r}\sin\left(n\pi \frac{\ln (r/a)}{\ln (b/a)}\right)	&\left(\frac{n\pi}{\ln (b/a)}\right)^2+\frac{1}{4}	&\left(\frac{1}{2}\ln \frac ba\right)^{-\frac 12} \\\hline
		\end{longtable}
		
		\subsubsection{(连带)勒让德多项式}
		\begin{longtable}[c]{c|*{4}{>{\rule[-0.5em]{0em}{2.5em}$\displaystyle}c<{$}}}
			$\bm L $	&\multicolumn{1}{c}{边界条件}	&\multicolumn{1}{c}{本征函数}	&\multicolumn{1}{c}{本征值}	&\multicolumn{1}{c}{归一化系数} \\\hline\endhead
			$\displaystyle \fdif{}{x}\left[(1-x^2)\fdif{}{x}\right]$	&\border{u}{x}{\pm 1}\mbox{有界}	& \fp{x}\footnote{\[l = 0,1,2,\cdots\],后同}	&l(l+1)	&\sqrt{\frac{2l+1}{2}} \\
			$\displaystyle \frac 1{\sin\theta}\fdif{}{\theta}\left(\sin\theta\fdif{}{\theta}\right)$	&\border{u}{\theta}{0,\pi}\mbox{有界}	& \fp{\cos \theta}	&l(l+1)	&\sqrt{\frac{2l+1}{2}} \\
			$\displaystyle \fdif{}{x}\left[(1-x^2)\fdif{}{x}\right] - \frac{m^2}{1-x^2}$	&\border{u}{x}{\pm 1}\mbox{有界}	&\mathrm{P}_l^m(x)	&l(l+1)	&\sqrt{\frac{(l+m)!}{(l-m)!}\frac{2l+1}{2}}\\
			$\displaystyle \frac 1{\sin\theta}\fdif{}{\theta}\left(\sin\theta\fdif{}{\theta}\right) - \frac{m^2}{\sin^2\theta}$	&\border{u}{\theta}{0,\pi}\mbox{有界}	& \mathrm{P}_l^m(\theta)	&l(l+1)	&\sqrt{\frac{(l+m)!}{(l-m)!}\frac{2l+1}{2}} \\\hline
		\end{longtable}
		
		\subsubsection{贝塞尔函数与球贝塞尔函数}
		\begin{longtable}[c]{c|*{4}{>{\rule[-0.5em]{0em}{2em}$\displaystyle}c<{$}}}
			$\bm L $	&\multicolumn{1}{c}{边界条件}	&\multicolumn{1}{c}{本征函数}	&\multicolumn{1}{c}{本征值}	&\multicolumn{1}{c}{归一化系数} \\\hline\endhead
			\multirow{4}{*}{\rule{0em}{5.5em}$\begin{aligned} \displaystyle \frac 1r\fdif{}{r}\left(r\fdif{}{r}\right) \\- \frac{\nu^2}{r^2}\end{aligned}$}
				&u|_{r=0}\mbox{有界},u|_{r=a}=0	&\fj{k_i r}\footnote{\[k_i\]满足边界条件,下同。}	&k_i^2	&\left[\frac {a^2}2\mathrm J_\nu'^2(k_ia)\right]^{-\frac 12} \\*
				&u|_{r=0}\mbox{有界},\left.\fdif ur\right|_{r=a}=0	&\fj{k_i r}	&k_i^2	&\left[\frac {a^2}2 \left(1 - \frac {m^2}{a^2 k_i^2}\right)\mathrm J_\nu^2(k_ia)\right]^{-\frac 12} \\*
				&u|_{r=0}\mbox{有界},\mbox{第三类}	&\fj{k_i r}	&k_i^2	&\left[\frac {a^2}2 \mathrm J_\nu'^2 + \frac {a^2}2 \left(1 - \frac {m^2}{a^2 k_i^2}\right)\mathrm J_\nu^2\right]^{-\frac 12} \\*
				&u|_{r=a,b}\mbox{的齐次条件}	&\mathrm J_\nu,\mathrm N_\nu	&k_i^2	&\mbox{略}\\\hline
			\multirow{3}{*}{\rule{0em}{3.5em}$\displaystyle \begin{aligned} \frac 1{r^2}\fdif{}{r}\left(r^2\fdif{}{r}\right) \\- \frac{l(l+1)}{r^2}\end{aligned}$}
				&u|_{r=0}\mbox{有界},u|_{r=a}=0	&\mathrm j_l(k_i r)	&k_i^2	&\left[\frac{a^3}{2}\mathrm j_l'^2(k_ia)\right]^{-\frac 12} \\*
				&u|_{r=0}\mbox{有界},u|_{r=a}	&\mathrm j_l(k_ir)	&k_i^2	&\left[\frac{a^3}{2}\left(j_l'^2 - j_lj_l''-\frac 1{k_ia}j_lj_l'\right)\right]^{-\frac 12}\\*
				&u|_{r=a,b}\mbox{的齐次条件}	&\mathrm j_l,\mathrm n_l	&k_i^2	&\mbox{略} \\\hline 
		\end{longtable}
		
		\subsubsection{二元本征值问题}
		归一化系数为1。
		\begin{longtable}[c]{c|*{3}{>{\rule[-0.5em]{0em}{2.5em}$\displaystyle}c<{$}}}
			$\bm L $	&\multicolumn{1}{c}{边界条件}	&\multicolumn{1}{c}{本征函数}	&\multicolumn{1}{c}{本征值}	\\\hline\endhead
			$\displaystyle \frac{1}{\sin \theta}\pard{}{\theta}\left[\sin \theta\pard{}{\theta}\right] + \frac{1}{\sin^2\theta}\pardsq{}{\phi}$	&\border{u}{\theta}{0,\pi}\mbox{有界},u|_\phi\mbox{周期}2\pi	& \mathrm{Y}_l^m(\theta,\phi)\footnote{\[m = 0,\pm 1,\pm 2,\cdots, \pm l\],\[2l+1\]重简并。权重因子\[\sin \theta\]。}	&l(l+1) \\\hline
		\end{longtable}
		
	\subsection{特殊函数}
	  \subsubsection{勒让德多项式(Legendre polynomials)}
		\begin{enumerate}
			\item \[l\]次勒让德多项式\[\fp{x}\]定义
			$$
		 	  \fp{x} = \sum_{n=0}^l \frac{1}{(n!)^2}\frac{(l+n)!}{(l-n)!}\left(\frac{x-1}2\right)^n \quad l = 0,1,2,\cdots
		 	$$
		 	\item 罗德里格公式(Rodrigues's formula)
		 	$$
		 	  \fp{x} = \frac{1}{2^l l!}\frac{\dif^l}{\dif x^l}\left[(x^2 - 1)^l\right]
		 	$$
		 	\item 另一个显明表达式
		 	$$
		 	  \fp{x} = \sum_{r=0}^{[l/2]}\frac{(-)^r(2l-2r)!}{2^lr!(l-r)!(l-2r)!}x^{l-2r}
		 	$$
		 	\item \[\fp{-x} =(-)^l\fp{x}\]
		 	\item \[\fp[2l]{0} = (-)^l\frac{(2l)!}{(2^l l!)^2},\quad \fp[2l+1]{0} = 0\]
		 	\item 零点均在\[(-1,1)\]
		 	\item 生成函数:
		 		$$
		 		  \frac{1}{\sqrt{1 - 2xt + t^2}} = \summ[l]{\infty} \fp{x}t^l \quad |t|<\left| x \pm \sqrt{x^2 -1}\right|
		 		$$
		 	\item 递推关系:
		 		\begin{eqnarray*}
		 		  (2l+1)x\fp{x} &=& (l+1)\fp[l+1]{x} + l\fp[l-1]{x} \\
		 		  \fp[l]{x} &=& \mathrm{P}'_{l+1}(x) - 2x\mathrm{P}'_{l}(x) + \mathrm{P}'_{l-1}(x) \\
		 		  \mathrm{P}'_{l+1}(x) &=& x\mathrm{P}'_{l}(x) + (l+1)\fp{x} \\
		 		  \mathrm{P}'_{l-1}(x) &=& x\mathrm{P}'_{l}(x) - l\fp{x}
		 		\end{eqnarray*}
		 	\item 前6次:
		 		$$
		 		  \begin{array}{*{3}{>{\rule{0em}{2em}\displaystyle }l}}
		 		  \fp[0]{x} = 1,	&\fp[1]{x} = x,	&\fp[2]{x} = \frac{1}{2}(3x^2 - 1), \\
		 		  \fp[3]{x} = \frac 12(5x^3-3x),	&\fp[4]{x} =\frac 18 (35x^4 - 30x^2 + 3),	&\fp[5]{x} = \frac 18 (63x^5 - 70x^3 + 15x) 
		 		  \end{array}
		 		$$
		\end{enumerate}
		
	  \subsubsection{连带勒让德函数(Associated Legendre polynomials)}
	    \begin{enumerate}
	    	\item 定义:\[m\]阶\[l\]次连带勒让德函数定义
	    		$$
	    		  \mathrm{P}_l^m(x) = (-)^m (1-x^2)^{m/2}\frac{\dif ^m}{\dif x^m}\fp{x}
	    		$$
	    	\item 相同阶但不同次的连带勒让德函数在区间\[[-1,1]\]上正交
	    	\item \[\mathrm{P}_l^{-m}(x) = (-)^m\frac{(l-m)!}{(l+m)!}\mathrm{P}_l^m(x)\]
	    	\item 正交性:
	    	\begin{eqnarray*}
	    	\int_{-1}^1\mathrm{P}_l^m(x)\mathrm{P}_{l'}^{-m}(x)\diff x &=& (-)^m\frac{2}{2l+1}\delta_{ll'} \\
	    	\int_{-1}^1\mathrm{P}_l^m(x)\mathrm{P}_{l}^{m'}(x)\frac{\diff x}{1-x^2} &=& \frac{1}{m}\frac{(l+m)!}{(l-m)!}\delta_{mm'} \\
	    	\int_{-1}^1\mathrm{P}_l^m(x)\mathrm{P}_{l'}^{-m'}(x)\frac{\diff x}{1-x^2} &=& \frac{(-)^m}{m}\delta_{mm'}
	    	\end{eqnarray*}
	    \end{enumerate}
	    
	  \subsubsection{球面调和函数(球谐函数)}
	    \begin{enumerate}
	     \item 归一化的球面调和函数定义(课件定义):
	     	$$
	     	  \mathrm{Y}_l^m(\theta,\phi) = \sqrt{\frac{2l+1}{4\pi}\frac{(l-m)!}{(l+m)!}}\mathrm{P}_l^m(\cos\theta)\e^{\mi m\phi}
	     	$$
	     	其中\[m = 0,\pm 1 , \pm 2 , \cdots ,\pm l\]
	     \item \[\mathrm Y_l^{m*}(\theta,\phi) = (-)^m\mathrm Y_l^{-m}(\theta,\phi)\]
	     \item 展开:
	      	  	\begin{numcases}{\frac 1 {|\vec r - \vec r'|} = }
	      	  		\frac 1{r'}\sum_{l=0}^\infty \sum_{m=-l}^l  \mathrm Y_l^{m*}(\theta,\phi)\mathrm Y_l^m(\theta',\phi')\left(\frac {r}{r'}\right)^l , & $r<r'$ \nonumber\\
	      	  		\frac 1{r}\sum_{l=0}^\infty \sum_{m=-l}^l \mathrm Y_l^{m*}(\theta,\phi)\mathrm Y_l^m(\theta',\phi')\left(\frac {r'}{r}\right)^l , & $r>r'$ \nonumber
	      	  	\end{numcases}
	     \item 加法公式:改变上式极轴方向可以得到
	     	$$
	     	  \fp{\cos\gamma} = \sum_{m=-l}^l \mathrm Y_l^{m*}(\theta,\phi)\mathrm Y_l^m(\theta',\phi')
	     	$$
	     	其中\[\cos \gamma = \cos \theta \cos \theta' + \sin \theta \sin \theta' \cos (\phi - \phi')\]
	    \end{enumerate}	      
	      
	  \subsubsection{贝塞尔函数(Bessel functions)和诺依曼函数(Neumann function)}
	   \begin{enumerate}
	     \item \[\nu\]阶贝塞尔函数\[\fj{z}\](第一类柱函数)定义
	     	$$
	     	  \fj{z} =\summ{0} \frac{(-)^n}{n! \Gamma(n + \nu + 1)}\left(\frac z2\right)^{2n+\nu} \quad |\arg z|<\pi
	     	$$
	     \item \[\nu\]阶诺依曼函数\[\fn{z}\](第二类柱函数)定义
	     	$$
	     	  \fn{z} = \frac{\cos \nu\pi \fj{z} - \fj[-\nu]{z}}{\sin \nu\pi} \quad |\arg z|<\pi
	     	$$
	     \item \[\fj[-n]{z} = (-)^n\fj[n]{z}\]其中\[n\]为整数
	     \item 线性相关性(朗斯基行列式):
	     	\begin{eqnarray*}
	     	 \Delta[\fj{z},\fj[-\nu]{z}] &=& -\frac{2}{\pi z}\sin \pi\nu \\
	     	 \Delta[\fj{z},\fn{z}] &=& \frac{2}{\pi z}
	     	\end{eqnarray*}
	     \item 诺依曼函数的整数极限:
	     	\begin{eqnarray*}
	     	 \fn[n]{z} &=& \lim\limits_{\nu\to n}\fn{z} \\
	 	  &=& \frac 2\pi \fj[n]{z}\ln\frac z2 - \frac 1\pi\sum_{k=0}^{n-1}\frac{(n-k-z)!}{k!}\left(\frac z2\right)^{2k-n} \\
	 	  &&\quad -\frac 1\pi\sum_{k=0}^{\infty}\frac{(-)^k}{k!(n+k)!}[\psi(n+k+1)+\psi(k+1)]\left(\frac z2\right)^{2k+n}\\
	     	\end{eqnarray*}
	     	当\[n = 0\]时,需去掉表达式中第二项的有限和
	     \item 递推公式
	     	\begin{eqnarray*}
	     	 \fdif{}{z}\left[z^\nu\fj[\nu]{z}\right] &=& z^\nu\fj[\nu -1]{z} \\
	     	 \fdif{}{z}\left[z^{-\nu}\fj[\nu]{z}\right] &=& -z^{-\nu}\fj[\nu +1]{z} \\
	     	 \fj[\nu-1]{z} - \fj[\nu+1]{z} &=& 2\mathrm J_\nu'(z) \\
	     	 \fj[\nu-1]{z} + \fj[\nu+1]{z} &=& \frac{2\nu}{z}\fj{z} \\
	     	 \left(\frac{\nu^2}{z^2} - 1\right)\fj{z} - \frac 1z\mathrm J_\nu'(z) &=& \mathrm J''_\nu(z)
	     	\end{eqnarray*}
	     	诺依曼函数形式完全相同。
	     \item \[\mathrm J_0'(z) = -\fj[1]{z}\]
	     \item \[\nu>-1\]或为整数时,\[\fj{x}\]有无穷多零点,零点都是实数且对称地分布在实轴上
	     \item 渐近展开
	     	\begin{eqnarray*}
	     	 \lim_{z\to 0}\fj{z} &\sim & \frac{1}{\Gamma(\nu+1)}\left(\frac z2\right)^\nu + O(z^{\nu+1})\\
	     	 \lim_{z\to 0}\fn{z} &\sim & -\frac{\Gamma(\nu)}{\pi}\left(\frac z2\right)^{-\nu} \quad \nu\neq 0\\
	     	 \lim_{z\to 0}\fn[0]{z} &\sim & \frac{2}{\pi}\ln \frac x2\\
	     	 \lim_{z\to\infty}\fj{z} &\sim & \sqrt{\frac 2{\pi z}}\cos\left(z-\frac{\nu\pi}2 - \frac \pi 4\right)\quad |\arg z|<\pi \\
	     	 \lim_{z\to\infty}\fn{z} &\sim & \sqrt{\frac 2{\pi z}}\sin\left(z-\frac{\nu\pi}2 - \frac \pi 4\right)\quad |\arg z|<\pi
	     	\end{eqnarray*}
	     \item 整数阶特有的性质:
	     	\begin{enumerate}
	     	 \item 生成函数
	     	 	$$
	     	 	  \exp\left[\frac z2\left(t - \frac 1t\right)\right] = \summ{-\infty} \fj[n]{z}t^n, \quad 0<|t|<\infty
	     	 	$$
	     	 	令\[t = \mi\e^{\mi\theta},z=kr\]得到平面波按柱面波展开式:
	     	 	$$
	     	 	  \e^{\mi kr\cos\theta} = \fj[0]{kr} + 2\summ{1}\mi^n\fj[n]{kr}\cos n\theta
	     	 	$$
	     	 \item 积分表示
	     	 	$$
	     	 	  \fj[n]{z} = \frac{1}{\pi}\int_0^\pi\cos(z\sin\theta - n\theta)\diff\theta
	     	 	$$
	     	 \item 加法公式
	     	 	$$
	     	 	  \fj[n]{x+y} = \sum_{k=-\infty}^\infty\fj[k]{x}\fj[n-k]{y}
	     	 	$$
	     	\end{enumerate}
	     \item 一些积分式:
	     	\begin{eqnarray*}
	     	 \int_0^\infty \e^{-ax}\fj[0]{bx} \diff x &=& \frac{1}{\sqrt{a^2+b^2}} \\
	     	 \left.\int_0^1(1-x^2)\fj[0]{\mu x}x\diff x\right|_{\fj[0]{\mu} = 0} &=& \frac{2}{\mu^2}\fj[2]{\mu} \frac{4}{\mu^3}\fj[1]{\mu}
	     	\end{eqnarray*}
	     \item 虚宗量贝塞尔函数
	   		\begin{enumerate}
	     	\item 第一类虚宗量贝塞尔函数定义
	     	\begin{eqnarray*}
	     	  \mathrm I_\nu(x) &=& \e^{-\mi\pi\nu/2}\fj{\mi x} \\
	     	    &=& \summ{0} \frac{1}{n!\Gamma(n+\nu+1)}\left(\frac x2\right)^{2n+\nu}
	     	\end{eqnarray*}
	     \item \[\mathrm I_n(x) = \mathrm I_{-n}(x)\]
	     \item 第二类虚宗量贝塞尔函数(McDonald function)定义
	     	\begin{eqnarray*}
	     	  \mathrm K_\nu (x) &=&\frac \pi{2\sin\nu\pi}\left[\mathrm I_{-\nu}(x) - \mathrm I_\nu(x)\right]\\
	     	  \mathrm K_n(x) &=& \lim_{\nu\to n}K_\nu(x) \\
	     	    &=&\frac 12\sum_{k=0}^{n-1}(-)^k\frac{(n-k-1)!}{k!}\left(\frac x2\right)^{2k-n} \\
	     	      &&\quad + (-)^{n+1}\sum_{k=0}^\infty \frac 1{k!(n+k)!}\left[\ln \frac x2 - \frac 12\psi(n+k+1) - \frac 12\psi(k+1)\right]\left(\frac x2\right)^{2k+n}
	     	\end{eqnarray*}
	     \item 渐近行为(约定\[\nu\ge 0\]):
	       \begin{eqnarray*}
	         \mathrm I_\nu(0) \mbox{有界} && \mathrm K_\nu(x)\mbox{无界}	\\
	         \lim_{x\to \infty} \mathrm I_\nu(x)\sim \sqrt{\frac 1{2\pi x}}\e^x &&
	         \lim_{x\to \infty} \mathrm K_\nu(x)\sim \sqrt{\frac \pi{2x}}\e^{-x}
	       \end{eqnarray*}
	   	   \end{enumerate}
	     \item 半奇数阶的贝塞尔函数
	     	\begin{eqnarray*}
	     	  \fj[1/2]{z} &=& \sqrt{\frac 2{\pi z}}\sin z \\
	     	  \fj[-1/2]{z} &=& \sqrt{\frac 2{\pi z}}\cos z \\
	     	  \fj[n+1/2]{z} &=& z^{n+1/2} \left(-\frac 1z\fdif{}{z}\right)^n\sqrt{\frac 2\pi}\frac{\sin z}{z} \\
	     	  \fj[-n+1/2]{z} &=& z^{n-1/2} \left(\frac 1z\fdif{}{z}\right)^n\sqrt{\frac 2\pi}\sin z \\
	     	  \fn[n+1/2]{z} &=&(-)^{n+1}\fj[-n-1/2]{z}
	     	\end{eqnarray*}
	   \end{enumerate}
	   
	   \subsubsection{汉克尔函数(Hankel function)}
	   又称第三类柱函数,第一种汉克尔函数\[\mathrm H_\nu^{(1)}(z)\]和第二种汉克尔函数\[\mathrm H_\nu^{(2)}(z)\]定义如下:
	   \begin{eqnarray*}
	    \mathrm H_\nu^{(1)}(z) &\equiv & \fj{z} + \mi\fn{z} \stackrel{z\to\infty}\sim \sqrt{\frac{2}{\pi z}}\exp\left[\mi\left(z-\frac{\nu\pi}2 - \frac{\pi}{4}\right)\right] \\
	    \mathrm H_\nu^{(2)}(z) &\equiv & \fj{z} - \mi\fn{z} \stackrel{z\to\infty}\sim \sqrt{\frac{2}{\pi z}}\exp\left[-\mi\left(z-\frac{\nu\pi}2 - \frac{\pi}{4}\right)\right]
	   \end{eqnarray*}
	   配合相应的时间因子\[\e^{-\mi\omega t}\]可以分别表示发散波和会聚波。
	   
	   \subsubsection{球贝塞尔函数(Spherical Bessel function)}
	   \begin{enumerate}
	     \item \[l\]阶球贝塞尔函数和\[l\]阶球诺依曼函数定义
	     \begin{eqnarray*}
	       y &=& c_0\mathrm j_l(z) + c_1\mathrm n_l(z) \\
	       \mathrm j_l(z) &=& \sqrt{\frac \pi{2z}}\fj[l+1/2]{z}\\
	         &=& \frac{\sqrt \pi}{2}\summ{0} \frac{(-)^n}{n!\Gamma(n+l+3/2)}\left(\frac z2\right)^{2n+l} \\
	       \mathrm n_l(z) &=& \sqrt{\frac \pi{2z}}\fn[l+1/2]{z} \\
	         &=& (-)^{l+1}\frac{\sqrt \pi}{2}\summ{0} \frac{(-)^n}{n!\Gamma(n-l+1/2)}\left(z2\right)^{2n-l-1}
	    \end{eqnarray*}
	    \item 前3阶
	    \newcommand{\rj}{\mathrm j}
	    \newcommand{\rn}{\mathrm n}
	    $$
		 	\begin{array}{*{3}{>{\rule{0em}{2em}\displaystyle }l}}
		 	 \rj_0(z) = \frac{\sin z}{z}, &\rj_1(z)=\frac{1}{z^2}(\sin z - z\cos z), &\rj_2(z) = \frac{1}{z^3}\left[(3-z^2)\sin z - 3z\cos z\right] \\
		 	 \rn_0(z) = -\frac{\cos z}{z}, &\rn_1(z) = -\frac{1}{z^2}(\cos z + z\sin z), &\rn_2(z) = -\frac{1}{z^3}\left[(3-z^2)\cos z + 3z\sin z\right]
		 	\end{array}
		$$
		\item 平面波按球面波展开
			$$
			  \e^{\mi kr\cos \theta} = \sum_{l=0}^{\infty}(2l+1)\mi^l\mathrm j_l(kr)\fp{\cos\theta}
			$$
		\item \[r=0\]处\[\rj_l(r)\]有界,\[\rn_l(r)\]无界
		\item \[r\to\infty\]渐进行为:
			\begin{eqnarray*}
			 \rj_l(r) &\sim & \frac{1}{r}\sin\left(r-\frac{l\pi}{2}\right)\\
			 \rn_l(r) &\sim & -\frac{1}{r}\cos\left(r-\frac{l\pi}{2}\right) 
			\end{eqnarray*}
	   \end{enumerate}
	   
\section{积分变换方法}
	\subsection{拉普拉斯变换(Laplace transform)}
	 \begin{enumerate}
	 \newcommand{\re}{\mathrm{Re}~}
	   \item 定义:
	   	$$
	   	  F(p) = \int_0^\infty\e^{-pt}f(t)\diff t
	   	$$
	   	\[F(p)\]称为\[f(t)\]的拉普拉斯换式,两者也分别称为像函数与原函数。\[\e^{-pt}\]是拉普拉斯变换的核,简写为:
	   	\begin{eqnarray*}
	   	 F(p) = \mathscr{L}\{f(t)\} &\mbox{或}& F(p)\risingdotseq f(t) \\
	   	 f(t) = \mathscr{L}^{-1}\{F(p)\} &\mbox{或}& f(t)\fallingdotseq F(p)
	   	\end{eqnarray*}
	   	注意:\[f(t)\]应该理解为\[t<0\]时\[f(t)=0\]
	   \item 变换存在的充分条件:
	   		\begin{enumerate}
	   		 \item \[f(t)\]在区间\[ [0 , \infty)\]中除了有限的第一类间断点外连续,而且有连续导数;
	   		 \item \[f(t)\]增长不超过指数函数,即:存在正数\[M>0\]及\[s'\ge 0\],使
	   		 	$$
	   		 	  \forall t. |f(t)|<M\e^{s't}
	   		 	$$
	   		 	\[s'\]的下界称为收敛横标,记为\[s_0\]
	   		\end{enumerate}
	   \item 拉普拉斯换式的极限
	   	\begin{eqnarray*}	   		
	   		\lim_{\re p\to+\infty} F(p) &=& 0 \\
	   		\lim_{\mathrm{Im}~p\to \pm\infty}F(p) &=& 0
	   	\end{eqnarray*}
	   \item 解析性:满足上面的条件,则\[F(p)\]在\[\re p\ge s_1>s_0\]的半平面中解析
	   \item 导数性质
	   	\begin{eqnarray*}
	   		f'(t)&\fallingdotseq & pF(p) - f(0)\\
	   		f^{(n)}(t) &\fallingdotseq & p^n F(p) - \sum_{k=1}^{n}p^{n-k}f^{(k-1)}(0)
	   	\end{eqnarray*}
	   \item 初值定理和终值定理
	   	\begin{eqnarray*}
	   	 \lim_{p\to+\infty}pF(p) &=& f(0)\\
	   	 \lim_{p\to 0}pF(p) &=& f(\infty)
	   	\end{eqnarray*}
	   \item 积分的拉普拉斯变换
	   	$$
	   	  \int_0^tf(\tau)\diff\tau \fallingdotseq \frac{F(p)}{p}
	   	$$
	   \item 像函数的导数反演
	   $$
	     F^{(n)}(p) \risingdotseq (-t)^nf(t)
	   $$
	   \item 像函数的积分反演
	   $$
	     \int_p^\infty F(q)\diff q \risingdotseq \frac{f(t)}{t}
	   $$
	   \[p=0\]时由定义可以得到:
	   $$
	     \int_0^\infty F(q)\diff q = \int_0^\infty \frac{f(t)}{t}\diff t
	   $$
	   \item 卷积定理
	   	$$
	   	  F_1(p)F_2(p) \risingdotseq \int_0^t f_1(\tau)f_2(t-\tau)\diff\tau
	   	$$
	   \item 普遍反演公式:
	    \begin{enumerate}
	     \item \[F(p)\]在区域\[\re p > s_0\]中解析,且\[|p|\to\infty\]时\[F(p)\]一致趋于0
	     \item 对于所有的\[\re p = s>s_0\],沿直线\[L:\re p = s\]的积分
	      $$
	        \int_{s-\mi\infty}^{s+\mi\infty}|F(p)|\diff\sigma
	      $$
	     收敛,则对于\[\re p = s>s_0\]
	     $$
	       F(p)\risingdotseq f(t) = \frac 1{2\pi\mi}\int_{s-\mi\infty}^{s+\mi\infty}F(p)\e^{pt}\diff p
	     $$
	    \end{enumerate}
	   \item 公式
	    $$
	      \frac{1}{\sqrt{p}}F(\sqrt{p})\risingdotseq \frac{1}{\sqrt{\pi t}}\int_0^\infty f(\tau)\e^{-\tau^2/4t}\diff\tau
	    $$
	   \item 变换表
	   	\begin{eqnarray*}
	   	  1 &\fallingdotseq & \frac 1p ,\quad \re p>0 \\
	   	  \e^{\alpha t} &\fallingdotseq & \frac{1}{p-\alpha} , \quad \re p>\re \alpha \\
	   	  \delta(t-\tau) &\fallingdotseq & \e^{-\tau p} \\
	   	  t &\fallingdotseq & \frac 1{p^2}\\
	   	  \frac 12 t^2 &\fallingdotseq & \frac{1}{p^3} \\
	   	  \frac 1{n!} t^n &\fallingdotseq & \frac 1{p^{n+1}}\\
	   	  \sin \omega t &\fallingdotseq & \frac{\omega}{p^2+\omega^2} \\
	   	  \cos \omega t &\fallingdotseq & \frac{p}{p^2+\omega^2} \\
	   	  \frac{\sin\omega t}{t} &\fallingdotseq & \frac \pi 2 - \arctan \frac p\omega \\
	   	  \mathrm{erfc}~ \frac{\alpha}{2\sqrt t} &\fallingdotseq & \frac 1p \e^{-\alpha\sqrt{p}}
	   	\end{eqnarray*}
	   	其中\[\mathrm{erfc}~x\]称为余误差函数,定义\[\mathrm{erfc}~x \equiv \frac{2}{\sqrt{\pi}}\int_x^\infty \e^{-\xi^2}\diff\xi\]。相关的还有误差函数\[\mathrm{erf}~x \equiv 1-\mathrm{erfc}~x = \frac{2}{\sqrt{\pi}}\int_0^x \e^{-\xi^2}\diff\xi\]
	 \end{enumerate}
	 
	\subsection{傅里叶变换(Fourier transform)}
	  \begin{enumerate}
	  	\item 定义
	  		\begin{enumerate}
	  		 \item 无界区间\[(-\infty,+\infty)\]上的函数\[f(x)\]在任意有限区间上只有有限个极值点和有限个第一类间断点
	  		 \item 积分
	  		 $$
	  		   \int_{-\infty}^{\infty}f(x)\diff x
	  		 $$
	  		 绝对收敛,则它的傅里叶变换存在
	  		 $$
	  		   F(k) = \mathscr{F}[f(x)] \equiv \frac 1{\sqrt{2\pi}}\int_{-\infty}^\infty f(x)\e^{-\mi kx}\diff x
	  		 $$
	  		 逆变换(反演)
	  		 $$
	  		   f(x) = \mathscr{F}^{-1}[F(k)]\equiv \frac 1{\sqrt{2\pi}}\int_{-\infty}^{\infty}F(k)\e^{\mi kx}\diff k
	  		 $$
	  		 \item 简记作\[f(x)\fallingdotseq F(k)\]
	  		\end{enumerate}
	  	\item 傅里叶变换可以看作是按正交完备集\[\left\{\frac{1}{\sqrt{2\pi}}\e^{\mi kx},k\in \mathbb R\right\}\]展开,实际上是无界区域的分离变量法。类似的方法可以把傅里叶变换推广到其他本征值问题中。
	  	\item 卷积定理
	  		\begin{eqnarray*}
	  		  F_1(k)F_2(k) &\risingdotseq & \frac 1{\sqrt{2\pi}}\int_{-\infty}^{\infty}f_1(\xi)f_2(x-\xi)\diff\xi \\
	  		  f_1(x)f_2(x) &\fallingdotseq & \frac 1{\sqrt{2\pi}}\int_{-\infty}^{\infty}F_1(\kappa)F_2(k-\kappa)\diff\kappa
	  		\end{eqnarray*}
	  	\item 导数公式
	  	$$f'(x)\fallingdotseq \mi kF(k),\quad F'(k)\risingdotseq -\mi xf(x)$$
	  	\item 积分的变换
	  	$$\int_{-\infty}^x f(\xi)\diff\xi\fallingdotseq \frac{F(k)}{\mi k},\quad \int_{-\infty}^k F(\kappa)\diff \kappa \risingdotseq -\frac{f(x)}{\mi x}$$
	  	\item  变换表
	  		\begin{eqnarray*}
	  		  1 &\fallingdotseq & \sqrt{2\pi}\delta(k) \\
	  		  \delta(x-x') &\fallingdotseq & \frac{\e^{-\mi kx'}}{\sqrt{2\pi}} \\
	  		  \e^{\mi k'x}&\fallingdotseq & \sqrt{2\pi}\delta(k-k') \\
	  		  \e^{\alpha x^2} &\fallingdotseq & \frac{1}{\sqrt{2\alpha}}\e^{-\frac{k^2}{4\alpha}}\\
	  		  \frac{1}{|x|} &\fallingdotseq & \frac{1}{\sqrt{|k|}} \\
	  		  \e^{a|t|} &\fallingdotseq & \sqrt{\frac{2}{\pi}}\frac{a}{a^2+k^2} \\
	  		  x^n &\fallingdotseq & \mi^n\sqrt{2\pi}\delta^{(n)}(k) \\
	  		  x^{-n} &\fallingdotseq & -\mi\sqrt{\frac{\pi}{2}}\frac{(-\mi k)^{n-1)}}{(n-1)!}\mathrm{sgn}(k)\\
	  		  \mathrm{sgn}(x) &\fallingdotseq & \sqrt{\frac 2\pi}\frac{1}{\mi k}
	  		\end{eqnarray*}
	  \end{enumerate}
	
\section{格林函数(Green's function)方法}
	\subsection{不含时的格林函数}	   
	  \begin{enumerate}
	    \item 格林第二公式
	    $$\iiint_V\left[u(\vec r)\nabla^2v(\vec r) - v(\vec r)\nabla^2u(\vec r)\right]\diff V = \oiint_{\partial V}\left[u(\vec r)\nabla(\vec r) - v(\vec r)\nabla u(\vec r)\right]\cdot \diff \vec S $$
	    \item 稳定问题的格林函数
	    	\begin{numcases}{}
	    	\left(\nabla^2 +k^2\right)u(\vec r) = -\rho(\vec r) &$\vec r \in V$\nonumber\\
	    	\left[\alpha u(\vec r)+\beta \pard{u(\vec r)}{\hat{\vec n}}\right]_{\Sigma} = f(\Sigma) &$\Sigma = \partial V$\nonumber
	    	\end{numcases}
	    	\[k=\rho = 0\]称为拉普拉斯方程,\[k=0\]称为泊松方程,\[\rho = 0\]称为亥姆霍兹方程。相应的格林函数满足的方程:
	    	\begin{eqnarray*}
	    	 \left(\nabla^2 +k^2\right) G(\vec r;\vec r') &=& -\delta(\vec r - \vec r') \\
	    	 \left[\alpha G+\beta \pard{G}{\hat{\vec n}}\right]_{\vec r\in\Sigma} &=& 0
	    	\end{eqnarray*}
	    	此时方程的解:
	    	$$
	    	  u(\vec r) = \iiint_{V}G(\vec r';\vec r)\rho(\vec r')\diff V' - \iint_{\Sigma}\frac {f(\Sigma')}{|\alpha|^2+|\beta|^2}\left[\alpha^*\frac{\partial G(\vec r';\vec r)}{\partial \hat{\vec n}'} - \beta^*G(\vec r';\vec r)\right] \diff \Sigma'
	    	$$
	    	其中\[\alpha,\beta\]可以是关于\[\sigma\]的函数。
	    	上面的式子实质是
	    	$$
	    	  u(\vec r) = \iiint_{V}G(\vec r';\vec r)\rho(\vec r')\diff V' - \iint_{\Sigma}\left[u(\vec r')\frac{\partial G(\vec r';\vec r)}{\partial \hat{\vec n}'} -  G(\vec r';\vec r)\frac{\partial u(\vec r')}{\partial \hat{\vec n}'}\right] \diff \Sigma'
	    	$$
	    	而根据边界条件,
	    	\begin{eqnarray*}
	    	 \alpha \left[u(\vec r')\frac{\partial G(\vec r';\vec r)}{\partial \hat{\vec n}'} -  G(\vec r';\vec r)\frac{\partial u(\vec r')}{\partial \hat{\vec n}'}\right] &=& f(\Sigma)\frac{\partial G(\vec r';\vec r)}{\partial \hat{\vec n}'} \\
	    	 \beta \left[u(\vec r')\frac{\partial G(\vec r';\vec r)}{\partial \hat{\vec n}'} -  G(\vec r';\vec r)\frac{\partial u(\vec r')}{\partial \hat{\vec n}'}\right] &=& -f(\Sigma)G(\vec r';\vec r)
	    	\end{eqnarray*}
	    \item 无界空间的解:
	    $$
	      G(\vec r ; \vec r') = \frac 1{4\pi}\frac{C\e^{\mi k|\vec r - \vec r'|}+(1-C)\e^{-\mi k|\vec r - \vec r'|}}{|\vec r - \vec r'|}
	    $$
	    系数与无穷远条件有关。通常表示为发散波即\[C = 1\]。\[C =0\]是会聚波。
	    \item 基尔霍夫公式(Kirchhoff's diffraction formula)
	    对于亥姆霍兹方程
	    \begin{eqnarray*}
	     \nabla^2 u(\vec r) + k^2 u(\vec r) = 0\\
	     u(\vec r)|_\Sigma ,\left.\pard{u(\vec r)}{\hat{\vec n}}\right|_{\Sigma}\mbox{已知}
	    \end{eqnarray*}
	    \[\Sigma\]外无界空间的解为(因为计算的是边缘之外的发散波,因而与上面有界范围的差一个符号):
	    $$
	      u(\vec r) = -\frac{1}{4\pi}\iint_\Sigma \frac{\e^{\mi k R}}{R}\left[u(\vec r')\left( \mi k - \frac{1}{R}\right)\frac {\vec R}R\cdot \hat{\vec n}+\pard{u(\vec r')}{\hat{\vec n}} \right]\diff\Sigma'
	    $$
	    其中\[\vec R = \vec r - \vec r',R = |\vec r - \vec r'|\]。
	    在\[\vec r'\]变化远小于\[r\]线度和变化,且\[R k \gg 1\],有:
	    \begin{eqnarray*}
	     \vec k &=& k\frac{\vec R}{R} \\
	     u(\vec r') &= & A(\vec r') \e^{\mi \vec k'\cdot\vec r'}\\
	     \nabla'u &\simeq & A(\vec r') \e^{\mi \vec k'\cdot\vec r'}\mi \vec k' \\	     
	     R &\simeq & R_0 - \frac{\vec k \cdot \vec r'}{k}
	    \end{eqnarray*}
	    代入得到
	    \begin{eqnarray*}
	       u(\vec r) &=& -\frac{\mi\e^{\mi kR}}{4\pi R}\iint_\Sigma A(\vec r') \e^{\mi(\vec k' - \vec k)\cdot \vec r'}(\vec k + \vec k')\cdot\diff\vec\Sigma' \\
	        &=& -\frac{\mi k\e^{\mi kR}}{4\pi R}\iint_\Sigma A(\vec r') \e^{\mi(\vec k' - \vec k)\cdot \vec r'}(\cos\theta + \cos\theta')\diff\Sigma'
	    \end{eqnarray*}
	    其中\[\cos\theta + \cos\theta'\]为倾斜因子
	    \item 广义格林函数\\
	    	对于泊松方程,\[\alpha = 0,\beta = 1\]时上面的格林函数满足的方程无解,因而需要引入广义的格林函数。
	    	\begin{eqnarray*}
	    	 \nabla^2 G(\vec r;\vec r') &=& -\left[\delta(\vec r - \vec r') - c(\vec r')u_0(\vec r)\right] \\
	    	 \left.\frac{\partial G(\vec r;\vec r')}{\partial \hat{\vec n}}\right| &=& 0
	    	\end{eqnarray*}
	    	其中\[u_0(\vec r)\]是方程
	    	\begin{eqnarray*}
	    	 \nabla^2 u_0(\vec r) &=& 0 \\
	    	 \left.\frac{\partial u_0(\vec r)}{\partial \hat{\vec n}}\right|_{\vec r \in \Sigma} &=& 0
	    	\end{eqnarray*}
	    	的非零解。\\
	    	因而有
	    	$$
	    	  c(\vec r') = \frac{u_0(\vec r')}{\iiint_V u_0^2(\vec r)\diff V}
	    	$$
	    	方程的解
	    	$$
	    	  u(\vec r) = \iiint_{V}G(\vec r';\vec r)\rho(\vec r')\diff V' + \iint_{\Sigma}f(\Sigma')G(\vec r';\vec r) \diff \Sigma' + c(\vec r)\iiint_{V} u(\vec r')u_0(\vec r')\diff V
	    	$$
	    	对于静电场,可以取\[u_0 = 1\],上式最后一项就是电势\[u(\vec r)\]在区域\[V\]内的平均值。
	    %\item 上面的泊松方程的格林函数方程可以定价地写成无界空间的泊松方程
	    %$$\nabla^2G(\vec r;\vec r') = -\left[\delta(\vec r - \vec r') + \sigma(\Sigma)\right]$$
	    %的形式,后一项可以解释为边界上的感生面电荷密度。此时有
	    %$$
	    %  G(\vec r;\vec r') = \frac{1}{4\pi}\frac 1{|\vec r - \vec r'|} + g(\vec r;\vec r')
	    %$$
	    %其中\[\nabla^2 g(\vec r;\vec r') = -\sigma(\Sigma)\]
	    \item 三维格林函数点源附近的行为:
	    $$
	      \lim_{\vec r \to \vec r'}G(\vec r;\vec r')\sim \frac 1{4\pi}\frac{\cos (k|\vec r - \vec r'|)}{|\vec r - \vec r'|}
	    $$%未能理解为什么可以用这种边界条件的情况……
	    二维泊松方程的情形
	    $$
	      G(x,y;x',y')\sim -\frac{1}{2\pi}\ln\sqrt{(x-x')^2+(y-y')^2}
	    $$
	   \item 格林函数的对称性\[G(\vec r';\vec r ) = G(\vec r ; \vec r')\](取决于具体方程,此处对于上述稳恒问题成立)
	   \item 用电像法求圆形区域的格林函数
	   	\begin{eqnarray*}
	   	 &&(\nabla_2)^2 G(\vec r;\vec r') = -\delta(\vec r - \vec r'),\quad |\vec r|,|\vec r'|<a \\
	   	 &&G(\vec r;\vec r')|_{r=a} = 0\\
	   	 &\Rightarrow & G(\vec r;\vec r') = -\frac{1}{2\pi}\left[\ln |\vec r- \vec r'| - \ln\left|\vec r - \left(\frac{a}{r'}\right)^2\vec r'\right|+\ln\frac{a}{r'}\right]
	   	\end{eqnarray*}
	  \end{enumerate}
	  
	\subsection{含时的格林函数}
	  \begin{enumerate}
	   \item 空间上的对称性与时间上的倒易性(取决于具体问题,此处对于波动方程和扩散问题成立)
	    $$
	      G(x',t';x,t) = G(x,-t;x',-t')
	    $$
	    在这个关系式中,将\[t\]和\[t'\]对换位置时出现的负号,正好保证了时间的先后次序不变,否则就会有悖于因果律的要求
	    \item 一维波动方程
	    	\begin{eqnarray*}
	    	 \pardsq{u(x,t)}{t} - a^2\pardsq{u(x,t)}{t} = f(x,t) &&0<x<l~t>0 \\
	    	 u(x,t)|_{x=0} = \mu(t),\quad u(x,t)|_{x=l} = \nu(t) &&t>0\\
	    	 u(x,t)|_{t=0} = \phi(x),\quad \left.\pard{u(x,t)}{t}\right|_{t=0} = \psi(x) &&0<x<l
	    	\end{eqnarray*}
	    	格林函数满足(先是\[G(x,-t;x',-t')\]满足的方程再利用对称性与倒易性得到):
	    	\begin{eqnarray*}
	    	 \pardsq{G(x',t';x,t)}{t} - a^2\pardsq{G(x',t';x,t)}{x} = \delta(x-x')\delta(t-t') &&0<x,x'<l~t,t'>0 \\
	    	 G(x',t';x,t)|_{x=0} = 0,\quad G(x',t';x,t)|_{x=l} = 0 &&t,t'>0\\
	    	 G(x',t';x,t)|_{t'<t} = 0,\quad \left.\pard{G(x',t';x,t)}{t}\right|_{t'<t} = 0 &&0<x,x'<l
	    	\end{eqnarray*}
	    	从而方程的解为
	    	\begin{eqnarray*}
	    	 u(x,t) &=& \int_0^l\diff x'\int_0^t Gf(x',t')\diff t' - \int_0^l\left[G\psi(x') - \phi(x')\pard{G}{t'}\right]_{t'=0}\diff x' \\
	    	 && \quad - a^2\int_0^t\left[\nu(t')\pard{G}{t'} - \mu(t')\pard{G}{x'}\right]_{x'=0}\diff t'
	    	\end{eqnarray*}
	    	其中
	    	\begin{eqnarray*}
	    	 G &=& G(x,t;x',t') \\
	    	  &=&\frac{2}{\pi a}\summ{1}\frac{1}{n}\sin\left(\frac{n\pi}{l}x'\right)\sin\left(\frac{n\pi}{l}x\right)\sin\left(\frac{n\pi}{l}a(t'-t)\right)\eta(t'-t)
	    	\end{eqnarray*}
	    \item 三维无界空间波动方程
	    	\begin{eqnarray*}
	    	  \left[\pardsq{}{t} - a^2\nabla^2\right]u(\vec r ,t) = f(\vec r,t),\quad t>0 \\
	    	  u(\vec r ,t)|_{t=0} = \phi(\vec r),\quad \left.\pard{u(\vec r ,t)}{t}\right|_{t=0} = \psi(\vec r)
	    	\end{eqnarray*}
	    	相应的格林函数满足的方程
	    	\begin{eqnarray*}
	    	  \left[\pardsq{}{t} - a^2\nabla^2\right]G(\vec r',t';\vec r,t) &=& \delta(\vec r - \vec r')\delta(t - t') \\
	    	  G(\vec r',t';\vec r,t)|_{t>t'} &=& 0 \\
	    	  \left.\pard{G(\vec r',t';\vec r,t)}{t}\right|_{t>t'} = 0
	    	\end{eqnarray*}
	    	经过傅里叶变换方法可以得到:
	    	$$
	    	  G(\vec r',t';\vec r,t) = \frac{1}{4\pi a}\frac{1}{|\vec r - \vec r'|}\delta(|\vec r - \vec r'| - a(t'-t))
	    	$$
	    	(\[\delta\]函数内体现了推迟势的物理意义)\\
	    	方程的解
	    	\begin{eqnarray*}
	    	  u(\vec r ,t) &=& \frac{1}{4\pi a^2}\iiint_{V=\{|\vec r'-\vec r|<at\}}\frac{f(r',t-|\vec r'-\vec r|/a)}{|\vec r'-\vec r|}\diff V \\
	    	  &&\quad + \frac{1}{4\pi a}\left[\iint_{\partial V}\frac{\psi(\vec r')}{|\vec r'-\vec r|}\diff\sigma' + \pard{}{t}\iint_{\partial V}\frac{\phi(\vec r')}{|\vec r'-\vec r|}\diff \Sigma'\right]
	    	\end{eqnarray*}
	  \end{enumerate}
	  
\section{其他}
	\subsection{行波解与驻波解的关系}
		在有限范围内分离变量得到的是驻波解。先做奇延拓再做周期延拓可以得到行波解。反之,行波解也可以通过驻波解叠加出来。
		
	\subsection{波}
		\begin{enumerate}
			\item 平面波:等相位面是平面的波,\[\psi = kx-\omega t\]
				\begin{eqnarray*}
				  &&\pardsq{u}{t} - a^2\pardsq{u}{x} = 0 \\
				  &\mbox{行波解(通解)}& u = f(x-at) + g(x+at)
				\end{eqnarray*}
			\item 柱面波:等相位面是柱面的波,\[\psi = kr-\omega t\]
				\begin{eqnarray*}
				&&\pardsq{u}{t} - \frac{a^2}{r}\pard{}{r}\left(r\pard{}{r}\right) = 0 \\
				&\Leftrightarrow &\pardsq{(\sqrt r u)}{t} - a^2\pardsq{(\sqrt r u)}{r} + \frac{\sqrt r u}{r^2}= 0 \\
				&r\to \infty \mbox{时}& \pardsq{(\sqrt r u)}{t} - a^2\pardsq{(\sqrt r u)}{r} = 0 \\
				&\mbox{近似行波解}& u \approx \frac 1{\sqrt r}[f(x-at) + g(x+at)]
				\end{eqnarray*}
			\item 球面波:等相位面是球面的波,\[\psi = kr-\omega t\]
				\begin{eqnarray*}
				&&\pardsq{u}{t} - \frac{a^2}{r^2}\pard{}{r}\left(r^2\pard{}{r}\right) = 0 \\
				&\Leftrightarrow &\pardsq{(ru)}{t} - a^2\pardsq{(ru)}{r} = 0 \\
				&\mbox{行波解(通解)}& u = \frac 1r[f(x-at) + g(x+at)]
				\end{eqnarray*}
				惠更斯原理与推迟势
		\end{enumerate}
	
	\subsection{泊松公式(Poisson formula)}
	  泊松方程的通解
		\begin{eqnarray*}
		&&\left\{\begin{array}{ll}
		\nabla^2 u = 0 &0<r<a,0<\phi <2\pi\\
		u|_{\phi = 0} =u|_{\phi = 2\pi} &0<r<a\\
		\left.\pard{u}{\phi}\right|_{\phi = 0} = \left.\pard{u}{\phi}\right|_{\phi = 2\pi} &0<r<a\\
		u|_{r=0}\mbox{有界} &0<\phi <2\pi \\
		u|_{r = a} = f(\phi) &0<\phi <2\pi
		\end{array}\right.\\
		&\Leftrightarrow& u = \frac{a^2 - r^2}{2\pi}\int_0^{2\pi}\frac{f(\phi')}{r^2 + a^2 - 2ar\cos(\phi - \phi')}\diff \phi'
		\end{eqnarray*}
		
	\subsection{勒让德展开公式}
		若\[w(z)\]和\[\varphi(z)\]在围道\[C\]上和内解析,\[z=a\]是\[C\]内一点,如果对于\[C\]上的任意点\[\xi\]都有参数\[t\]满足\[|t\varphi(\xi)|<|\xi - a|\],则:
		\begin{enumerate}
			\item 方程\[z=a+t\varphi(z)\]在\[C\]内有且只有一根
			\item 函数\[w(z)\]可以按照\[t\]展开成幂级数:
				$$
				  w(z) = w(z) + \summ{1}\frac{t^n}{n!}\frac{\dif^{n-1}}{\dif a^{n-1}}\left\{w'(a)[\varphi(a)]^n\right\}
				$$
		\end{enumerate}
		
		令\[\varphi(z) = (z^2-1)/2\],且\[z = x+t\varphi(z)\],可以解出
		$$
		  z = \frac{1}{t}\left(1-\sqrt{1-2tx+t^2}\right)
		$$
		利用上面的展开公式,两边同时对\[x\]求导得到~Rodrigues~公式。
\end{document}
